\documentclass[class=article, crop=false]{standalone}
\usepackage[subpreambles=true]{standalone}
\usepackage{import}
\usepackage{preamble}
\usepackage{pdfpages}
\begin{document}
\section{Conclusions}
\label{sec:Conclusions}
This paper has contributed to the literature by pooling a large, clean dataset with a large proportion of ethnic minorities, spanning 23 years. The analysis has shown that pay gaps are large and persistent over time. Decompositions corroborate the existing literature, showing that labour market outcomes are starkly different not only for males and females but also for ethnicities within genders. Because the majority of the female wage gap was explained and not unexplained, it indicates that differing labour market histories have been accounted for by the number of children, and that the results are reliable. Consequently, this paper has given a unique insight into the evolution of the wage gap over time, not just for men, but also for women.

Black, Asian and mixed race males find themselves earning a lower wage than their White counterparts, with only Chinese males not experiencing a wage gap. Black and Asian male wages are mostly depressed by occupational clustering leading to a negative wage gap, despite living in regions which typically earn higher wages. Mixed race males suffer from lower levels of experience and tenure, depressing their wages. Chinese males' higher levels of education results in them earning same as their White counterparts. Chinese and mixed race males have no evidence of outright pay discrimination, unlike Black and Asian males who have half of their wage gap unexplained. This is possible evidence of pay discrimination, despite Economic theory suggesting that is not possible without barriers to entry in markets, or collusion between firms. For women, there is no evidence of pay discrimination. However, all female ethnic groups experience lower pay than men. Ethnic minority females typically cluster in higher paying occupations than White women, and enjoy higher levels of education, along with higher pay.

Black and Asian workers have struggled since the financial crisis. Mixed race workers along with Black and Asian males saw some evidence of pay discrimination in the immediately preceding years. Any other variation is likely to be due to the endowments of the sample.

Endowments have a profound impact on wages, with this paper finding that occupational clustering is a key factor affecting wages. This negatively affects men, but positively influences the wages of females. Regardless of the direction of the impact on wages, clustering is possible evidence for discrimination, as minorities have a tendency to avoid particular occupations and favour others. Given that ethnic minority males have higher levels of education yet cluster in lower paid occupations, it is important for future research to explore the endogienity of education and occupational decision making. It is also important to note that the \enquote{Explained} components are not necessarily justifiable, and often are a direct cause of failures in government policy. These components serve to detail why pay gaps exist, and facilitate policy changes in light of the above conclusions.

The disparities in pay extend beyond pay, with minorities being more likely to be unemployed or inactive. Self selection away from industries may contribute to this, and will also serve to exacerbate future pay inequalities. Unless clustering into industries is addressed these disparities are likely to continue or worsen as new generations of minorities make decisions regarding their education and location in the UK.

Well designed policies can improve lead to a Pareto Improvement, bring both equity and efficiency gains \citep{LundbergB}. Future policy will need to focus on providing ethnic minorities with education which can enable them to enter more highly paid occupations. \textcolor{red}{More theoretical policy suggestions in \citep{LundbergB}.} Discrimination is certainly not absent from the UK labour market and, if anything, has worsened over the past 23 years. There is a need for policy to address to the significant pay disparities experienced by ethnic minorities even today.

\textcolor{red}{
If the endogienity is an issue then affirmative policies could be used to subsidise employment success of minorities. This will help to overcome the lack of investment in human capital caused by employer prejudices \citep{Coate}. Moreover this will address the overriding issue of ethnic employment targets, which work to further negative stereotypes if minorities have already under-invested in human capital, especially if quotas create shortages \citep{Welch}. Move this to a policy section? Or to the intro and then put the suggested policy in the conclusion?
}
\ifstandalone
\bibliography{essaybib}
\fi
\end{document}