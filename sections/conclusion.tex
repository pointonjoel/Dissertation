\documentclass[class=article, crop=false]{standalone}
\usepackage[subpreambles=true]{standalone}
\usepackage{import}
\usepackage{preamble}
\usepackage{pdfpages}
\begin{document}
\section{Conclusions}
\label{sec:Conclusions}
This paper has contributed to the literature by pooling a large, clean dataset with a large number of ethnic minorities, spanning 23 years. The analysis has shown that pay gaps are large and persistent over time. Decompositions corroborate the existing literature, showing that labour market outcomes are starkly different for male and female ethnic minorities. The majority of the female wage gap is explained and not unexplained, indicating that differing labour market histories have been accounted for by the number of children and that the results are reliable. Consequently, this paper has given a unique insight into the evolution of the wage gap over time, not just for men, but also for women. %also find education to be a key factor in reducing the pay gap, similar to Longhi2.

Pay discrimination is most common among Black and Asian workers of both genders. Chinese and mixed-race workers have little (if any) evidence of pay discrimination, with males having no statistically significant pay gap compared to White males, and females having a positive wage gap thanks to higher levels of education and occupational clustering. The \enquote{unexplained} pay gap is largest for men, with women facing less, if any, evidence of direct pay discrimination. However, all women are subject to the gender pay gap.
%All ethnic minorities have favourable characteristics, but for men, the \enquote{unexplained} component outweighs this. For women, education and occupational clustering outweigh any discrimination they face. Consequently, the ethnic wage gap for males favours White workers, whereas, for women, it favours non-White workers.

Numerous characteristics have a large influence on the pay of ethnic minorities. All minorities have higher levels of education relative to their White counterparts. However, occupational clustering negatively impacts the wages of men, but positively influences female wages. Regardless of the direction of the impact on wages, clustering is possible evidence for discrimination, as minorities tend to avoid particular occupations and favour others. Given that ethnic minority males have higher levels of education yet cluster in lower-paid occupations, future research needs to explore the endogeneity of education and occupational decision making. Other \enquote{explained} components may also be vectors of structural discrimination, and more research into these is required.

The \enquote{unexplained} pay gap can be thought of as an upper bound on the level of pay discrimination in the labour market. \enquote{Unexplained} pay gaps are largest for men, but there is also evidence of direct pay discrimination against Black and Asian women. No clear trends can be observed in the \enquote{unexplained} component of the pay gap. There appears to have been a slight improvement since 2011/2012, but this is from an increase in the \enquote{unexplained} pay gap in the wake of the financial crisis, indicating that discrimination increased during this period, and has subsequently fallen.

Discrimination extends beyond pay gaps, with minorities being more likely to be unemployed or inactive. Well designed policies that address significant disparities experienced by ethnic minorities can lead to a Pareto Improvement, bringing both equity and efficiency gains \citep{LundbergB}. Future policy will need to focus on reducing the barriers preventing ethnic minority males into highly paid occupations, likely caused by structural racism. Discrimination is certainly not absent from the UK labour market, and policy must continue to dismantle the entrenched effects of structural discrimination. %\textcolor{red}{More theoretical policy suggestions in \citep{LundbergB}.} 

% \textcolor{red}{
% If the endogienity is an issue then affirmative policies could be used to subsidise employment success of minorities. This will help to overcome the lack of investment in human capital caused by employer prejudices \citep{Coate}. Moreover this will address the overriding issue of ethnic employment targets, which work to further negative stereotypes if minorities have already under-invested in human capital, especially if quotas create shortages \citep{Welch}. Move this to a policy section? Or to the intro and then put the suggested policy in the conclusion?
% }
\ifstandalone
\bibliography{essaybib}
\fi
\end{document}