\documentclass[class=article, crop=false]{standalone}
\usepackage[subpreambles=true]{standalone}
\usepackage{import}
\usepackage{preamble}
\usepackage{pdfpages}
\begin{document}
\section{Conclusions}
\label{sec:Conclusions}
This paper has contributed to the literature by pooling a large, clean dataset with a large proportion of ethnic minorities, spanning 23 years. The analysis has shown that pay gaps are large and persistent over time. Decompositions corroborate the existing literature, showing that labour market outcomes are starkly different not only for males and females but also for ethnicities within genders. Because the majority of the female wage gap was explained and not unexplained, it indicates that differing labour market histories have been accounted for by the number of children and that the results are reliable. Consequently, this paper has given a unique insight into the evolution of the wage gap over time, not just for men, but also for women. It has also shown that the disparities in pay extend beyond pay, with minorities being more likely to be unemployed or inactive.

Pay discrimination is most common among Black and Asian workers, both male and female. Chinese and mixed race workers have little (if any) evidence of pay discrimination, with males having no statistically significant pay gap compared to White males, and females having a positive wage gap thanks to higher levels of education and occupational clustering. All ethnic minorities have favourable characteristics, but for men, the level of discrimination outweighs this. For women, their characteristics outweigh the discrimination they face. Consequently, the ethnic wage gap for males favours White workers, whereas, for females, it favours non-White workers.

Occupational clustering is a key factor affecting wages, negatively affecting men, but positively influencing females. Regardless of the direction of the impact on wages, clustering is possible evidence for discrimination, as minorities tend to avoid particular occupations and favour others. Given that ethnic minority males have higher levels of education yet cluster in lower-paid occupations, future research needs to explore the endogeneity of education and occupational decision making. Other \enquote{Explained} components are also not necessarily justifiable, and often are a direct cause of failures in government policy.

No clear trends can be observed in the \enquote{unexplained} component, both with males and females. Overall there seems to have been a slight improvement since 2011/2012, but this is from an increase in the \enquote{unexplained} pay gap in the wake of the financial crisis, indicating that discrimination increased during this period, and has subsequently fallen.

Discrimination is certainly not absent from the UK labour market and, if anything, has worsened over the past 23 years. There is a need for policy to address the significant pay disparities experienced by ethnic minorities even today. Well designed policies can improve lead to a Pareto Improvement, bring both equity and efficiency gains \citep{LundbergB}. Future policy will need to focus on providing ethnic minorities with education that can enable them to enter more highly paid occupations. \textcolor{red}{More theoretical policy suggestions in \citep{LundbergB}.} 

\textcolor{red}{
If the endogienity is an issue then affirmative policies could be used to subsidise employment success of minorities. This will help to overcome the lack of investment in human capital caused by employer prejudices \citep{Coate}. Moreover this will address the overriding issue of ethnic employment targets, which work to further negative stereotypes if minorities have already under-invested in human capital, especially if quotas create shortages \citep{Welch}. Move this to a policy section? Or to the intro and then put the suggested policy in the conclusion?
}
\ifstandalone
\bibliography{essaybib}
\fi
\end{document}