\documentclass[class=article, crop=false]{standalone}
\usepackage[subpreambles=true]{standalone}
\usepackage{import}
\usepackage{preamble}
\usepackage{pdfpages}
\begin{document}
\section{Data Source}
\label{sec:Data}
In order to conduct a rigorous analysis of the \acrshort{uk} labour market, a large sample size is required. Initially, the \acrfull{bcs} 1970 was preferred, but the sample of ethnic minority groups was insufficient. Additionally, wage data within the \acrshort{bcs} is limited, causing the overall sample from ethnic minorities with wage data to be very small. Consequently, the \acrfull{lfs} is the main source of data for subsequent analysis \citep{ONSk}.

The \acrshort{lfs} was initially a bi-annual survey of \acrshort{uk} households, first conducted in 1973. In 1984 it became an annual survey, and since 1992 it has been conducted quarterly. The \acrshort{lfs} is the largest of its kind in the \acrshort{uk} and used for official \acrshort{uk} statistics regarding both employment and unemployment \citep{ONS}. It covers approximately 40,000 households and 100,000 individuals in each quarter, with each household being interviewed for 5 waves (quarters). The period covered will be 1997 through to 2019, inclusive. Reliable earnings data, not subject to attrition bias, is only available from 1997 \citep{ONSa} and, at the time of writing, only three-quarters of the 2020 data are available, resulting in a reduced sample size. Only those in wave 1 have been kept to remove attrition bias and because changes in wages from wave 1 to 5 is expected to be minimal (which would generate serial-correlation from clustering).

%Additionally, there are biases in responses due to social distancing and removal of face-to-face interviews; whilst weights have been adjusted, the analysis in this paper has not used (income) weights \citep{ONSb}. 

There are several drawbacks to using the \acrshort{lfs}. Firstly, it does not include past labour market experience, meaning that detailed analysis of periods out of the labour market cannot be conducted. Secondly, 2001-Q1 is a corrupted file on the \acrshort{uk} Data Service, and so is excluded from the analysis, consistent with \citet{Longhi}. Finally, there is a high proportion of proxy responses, which can bias the likelihood of reporting certain qualities (\citet{Clarke} and \citet{Davies}). However, the effect on this paper's conclusions is believed to be minimal. Despite these drawbacks, the \acrshort{lfs} is the richest data source of its kind, with a large sample size and the ability to conduct analysis across time.

To enrich the dataset, regional unemployment data was pooled from \citet{ONSf} with data regarding British attitudes coming from the \acrfull{us} survey, which was available in waves 1, 3 and 6 \citep{US}. This was merged with the \acrshort{lfs} dataset and all forthcoming analysis uses this merged dataset unless otherwise stated. In total, 12 million observations are pooled which, when reduced to those who were employed in wave 1, reduces the sample to approximately 1 million. Keeping only observations with data for both hourly wages and each of the explanatory variables reduces the sample size to 515,980. This is comprised of 473,485 (93.44\%) White workers, 9,430 who are Black (1.86\%), 1,712 Chinese (0.34\%), 18,891 who are  Asian but not Chinese (3.73\%), and 3,221 (0.64\%) who are mixed-race. No other ethnic minorities are considered in this paper.

For the remainder of the paper, the ethnic groupings are defined as in Figure \ref{fig:ethnicity_definitions}:
\begin{figure}[h!]
    \begin{center}
        \begin{tabular}{l*{2}{c}}
            \hline\hline
            \textbf{Ethnicity} & \textbf{Definition} \\
            \hline
            White & \makecell{White \\(\textit{incl. British})} \\
            Black & \makecell{Black \\(\textit{incl. British/Caribbean/African})} \\
            Chinese & Chinese \\
            Asian & \makecell{Asian excl. Chinese \\(\textit{incl. British/Indian/Pakistani/Bangladeshi})} \\
            Mixed & \makecell{Mixed-race \\(\textit{incl. Black Mixed})} \\
            \hline\hline
        \end{tabular}
    \end{center}
    \caption{Ethnicity definitions used throughout this paper}
    \label{fig:ethnicity_definitions}
\end{figure}

\subsection{Setup}
\label{sec:Setup}
The forthcoming analysis uses hourly wages, to eliminate the differences in weekly hours worked and the income effect of increased hourly wages. This increases accuracy but loses 12.6\% of the sample compared to weekly wages. Net wages are more likely to be accurately reported, but gross wages have been used to remove any effects of tax changes. Wages have also been deflated using the GDP deflator \citep{ONSh} so that controlling for changes over time more accurately depicts any changes in the labour market which affect minorities in differential ways. 

To approximate experience, 'age minus years of education' has been used. This is relatively accurate for men, who are less likely to take time out of the labour market or be employed part-time \citep{Olsen}. However, there is a scarring effect of prolonged non-full-time work (contributing to 52\% of the simulated pay gap), which is likely to have biased my findings for women\footnote{See further discussion by \citet{Walby}.}. Any differences in continuous employment across ethnicities will therefore be encompassed within the \enquote{unexplained} portion.

Numerous regional controls have been employed. The number of years living in the \acrshort{uk} will control for differing outcomes seen with non-natives. This is simply the age for natives (49\% of minorities in the sample), but for immigrants, it uses the year of arrival in Britain, combined with the age and year of interview. Thus there can be inaccuracies depending on the months of interview, arrival, and respondent’s birthday. These issues are thought to be relatively insignificant and are the same as those faced by \citet{Berthoud}. Within the \acrshort{uk}, all regions except Northern Ireland have been included, due to minorities clustering in London and the South East, with only 0.6\% of the sample living in Northern Ireland. Additionally, the ethnicity question is not comparable between Northern Ireland and the rest of the \acrshort{uk} \citep{Longhi}.

Due to the dataset covering 23 years, there are several challenges in consistently pooling the data. Most variables were relatively unchanged from 2001 onwards, notable exceptions include industry and occupation data, with industry data using an approximate conversion to the 1992 classification. Education data collected by the \acrshort{lfs} has changed over time, so we have condensed our controls into 4 categories: under 5 GCSEs at grade C, 5 GCSEs at grade C or above, 2 or more A-Levels (or equivalent), an undergraduate degree. Furthermore, a unique household identifier is only included in Stata datasets from 2001-Q3 onwards. Consequently, all analysis will be conducted without clustering, to keep 1997-2001 within the analysis. The effect of this on the reliability of our findings will be discussed in the 'limitations' section (Section \ref{sec:limitations}).

% Years 1992-1996 were initially considered, but later dropped because many variables substantially differed from more recent years, some key wage or education data was poor or missing, and earnings questions before 1997 were commonly unanswered. 

\section{Descriptive Analysis}
\label{sec:Descriptive Analysis}
\subsection{Descriptive Statistics}
\label{sec:Descriptive Statistics}
Descriptive Statistics have been included in Table \ref{tab:sumstats} of \hyperref[sec:appendixA]{Appendix A}. Given that the \acrshort{lfs} is a large sample size, and we have pooled data from 23 years, it is expected to be representative of the wider \acrshort{uk} population. 26\% of the sample work part-time, 28\% of the sample work in the public sector, and 52\% of the sample is female. 71\% of the sample are married, and the average gross hourly pay is £15.77 per hour, using 2018/19 prices. However, there is significant variation in these figures for each ethnic group considered, which the next section outlines.

\subsection{Labour Market Outcomes}
\label{sec:Labour Market Outcomes}
It is important to first understand the wider labour market conditions in the sample. Figure \ref{fig:u_rates} depicts the unemployment rates for each ethnicity, with Whites being the least likely to be unemployed. Only Chinese workers had a lower unemployment rate at times over the sample period. Note that Black and mixed-race workers had the highest unemployment rates over the entirety of the sample, although all minorities saw unemployment rates fall over the sample period.
\begin{figure}[h]
\begin{subfigure}{0.5\textwidth}
\centering
    \title{Unemployment rates (age 16-64)}
    \import{graphs/}{u_rates}
    \caption{Unemployment rate over time}
    \label{fig:u_rates}
\end{subfigure}
%\hspace{35pt}
%\flushright
\begin{subfigure}{0.5\textwidth}
\centering
    \title{Economic Activity by Ethnicity (age 16-64)}
    \import{graphs/}{employment_stacked}
    \caption{Labour market status (average of sample)}
    \label{fig:employment_stacked}
\end{subfigure}
\caption{Labour market characteristics, by ethnicity}
\label{fig:labour_market}
\end{figure}

One of the reasons for the relatively low Chinese unemployment rate is the high proportion of inactive workers. Figure \ref{fig:employment_stacked} shows that 37.4\% of the Chinese sample were inactive on average over the sample period. This may be due to workers becoming discouraged, or cultural norms, given that the female inactivity rate was 42.1\% compared to 31.9\% for men. Additionally, 10.5\% were self-employed, the highest of any ethnic grouping, possibly explaining the low unemployment rate.

The heterogeneity between Chinese and Asian (excl. Chinese) labour market participants is important to note. Both have similar labour market characteristics, but the reasons for these are starkly different. Chinese workers are much less likely to have 2 or more children (17.0\% compared to 32.7\%). Additionally, Asian women have stubbornly high inactivity rates in the 30-39 age bracket relative to 16-21, with the ratio of the two falling to just 0.65, compared to 0.45-0.55 for most other ethnicities. This, combined with the greater number of children, on average, is evidence of heterogeneity in labour market histories for women (possibly due to discouragement during teenage years and early twenties). Consequently, the effect the number of children on wages will be contained in the \enquote{explained} part of the analysis, with any behaviours in response to this which differ by ethnicity will bias the \enquote{unexplained} portion upwards.

Furthermore, Figure \ref{fig:employment_stacked} shows that Black and mixed-race workers both had some of the lowest inactivity rates, yet high unemployment rates. This may be due to a willingness to work, but difficulty in obtaining employment. It is also notable how both groups had the lowest self-employment rates among the sample. Consequently, the following analysis is likely to have underestimated the pay gap, given that the \acrshort{lfs} contains no self-employed wage data, which is typically lower than median wages \citep{GOVf}.

% Structural unemployment in areas that house clusters of minorities could be a factor causing the higher levels of inactivity and unemployment (\citet{Platt} and \citet{Berthoud}). Whilst this varies for males relative to females, minorities are consistently underrepresented in the labour market, particularly Chinese males and Asian women \citep{Longhi2}. Figure \ref{fig:population_by_ethnicity} shows how there is significant clustering in areas of the \acrshort{uk}. Affluent areas have a low Black population, whereas metropolitan and cosmopolitan areas have much higher levels. This may explain why minority ethnic groups face higher levels of unemployment, possibly structurally, due to clustering geographical areas. Any such clustering is also likely to affect wages, through occupational clustering, a possible mechanism for the endogeneity explored in Section \ref{sec:Literature}.

% \begin{figure}[]
% \centering
%     \title{Population by ethnicity}
%     \import{graphs/}{population_by_ethnicity}
%     \caption{Percentage of population living in areas of England by ethnicity at 2016 Census. Author's calculations from \citep{ONSd}}
%     \label{fig:population_by_ethnicity}
% \end{figure}

\subsection{Raw Wages}
\label{sec:Wages}
An initial view of median hourly wages reveals vast disparities. Figure \ref{fig:hourly_percentiles} contains the pay gap at specific percentiles, showing it is much bigger at the higher end of the distribution. Black, male workers particularly struggle to reach high paying jobs, with Black females facing similar challenges, but to a lesser extent. The lowest 10\% of Black females earn £0.69 more per hour than White females. Across the board, the distribution is much more even for women, with White females earning one of the lowest hourly wages. Significant heterogeneity within ethnic groups is observed (particularly for Chinese workers) --- however, analysis at different points of the distribution is beyond the scope of this paper. Although, this can be seen with the lowest 10\% of Chinese male workers being significantly disadvantaged, earning just £6.13 per hour, yet the highest 10\% earn more than any other ethnic grouping. The converse is true for Black females, who have much less variation in earnings compared to other women. Note that this table corroborates the existing literature regarding gender pay gaps, with women earning less than men at each percentile.

\setstretch{1.25}
\import{tables/}{hourly_percentiles}
\doublespacing

Even though similar proportions of the working population work full-time relative to part-time for each minority (24-27\%), the earnings profiles in Figure \ref{fig:earnings_profiles} reveal important features of the labour market. Among full-time earners, women earn a higher hourly wage in the 16-17 age bracket, and similar amounts until the age of 40. However, beyond the age of 40, the reversal in earnings is much more pronounced than for males. This could be due to the reintroduction of mothers into the labour market who accept a lower wage, skewing the median wage downwards. Interestingly, this effect is less pronounced among part-time workers, even when expressed in proportions. This could be because part-time workers, the majority of whom are women, are less likely to receive pay rises and are more concentrated near the minimum wage (\citet{IFS} and \citet{ONSi}).

\begin{figure}[h]
\centering
    \title{Earnings Profiles by age}
    \import{graphs/}{earnings_profiles}
    \caption{The set of earnings profiles in 2020, derived from \citet{ONSc}. For additional analysis including education, see \citet{Mincer}}
    \label{fig:earnings_profiles}
    \vspace{-10pt}
\end{figure}

Significant changes in wages have occurred over time. Figure \ref{fig:median_wages} shows how median hourly wages have been increasing for each ethnic group\footnote{The Chinese and mixed-race time series are significantly more volatile, which is likely due to the smaller sample size noted in Section \ref{sec:Data}. Median wages minimise the effect of outliers from year to year.}, with the difference between each line giving the relative pay gap. Asian workers have consistently been paid less than Whites, with those who are Black initially earning the same as those who are White, but earning less after the financial crisis of 2008. Both the Asian and Black pay gap has narrowed in the last three years. Chinese workers have earned more than Whites, relatively consistently, but have accelerated in earnings growth since the financial crisis, with a dip in recent years. Finally, mixed-race workers have earned similar amounts to White employees, with some periods above and some periods below the median White wage. The trends post-financial crisis have been increasingly heterogeneous, which may be due to a combination of increased migration \citep{ONSj} and a reduction in unemployment rates, particularly for minorities (Figure \ref{fig:labour_market}). Exploring why these earnings differentials exist is the subject of the preceding chapters; it may be due to different characteristics, or outright pay discrimination.

\begin{figure}[]
\centering
    \title{Median wages}
    \import{graphs/}{median_wages}
    \caption{Median wages over time for each ethnicity, 2018/19 prices}
    \label{fig:median_wages}
\end{figure}

% \textcolor{red}{compare sample to the \acrshort{uk} population --- by gender, ethnicity and SIC2D. Also look at the dataset and likelihood of being married and having a degree etc --- different levels of education, as this might fill some of the research that was recommended in \citep{Longhi2} on page 81 (101 for adobie pdf viewer).
% This can be symptomatic of self-selection into industries, as seen with the equilibria of signalling models (or tase-based ones??).
% Use \citep{Shields} when talking about sample selection and heckman correction. They also mention the importance of immigrants.
% Get employment rates conditional on education, by ethnicity.
% Occupational segregation over time --- \citep{Longhi3} finds it has fallen.
% Swap occup to 2 digit and industry to 1 digit.}

%-----------------------------------------------------

\begin{comment}
\begin{enumerate}
    \item Tabulate the number of people in each 2-digit SIC92 for each ethnicity used.
    \item Blinder-Oaxaca decomposition.
    \item Pseudo-panel data??
    \item Clustering --- but can only happen for 2001 onwards
    \item Head of household --- only keep the head of the household, but mostly men (show stats to prove this! AND show if and when the definition changes)
\end{enumerate}
Issues with ommited variables in the oaxaca decomps --- are they consistent?

\end{comment}

\begin{comment}
\import{tables/}{occup_male}
\import{tables/}{occup_female}
\newpage
\import{graphs/}{unexplained_male_stacked} %NEED TO UPDATE EXCEL
\newpage
\import{graphs/}{explained_female_stacked}
\newpage
\import{graphs/}{explained_male_stacked}
\newpage
\import{graphs/}{unexplained_female_stacked}
\end{comment}


\ifstandalone
\bibliography{essaybib}
\fi
\end{document}