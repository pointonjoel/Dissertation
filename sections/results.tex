\documentclass[class=article, crop=false]{standalone}
\usepackage[subpreambles=true]{standalone}
\usepackage{import}
\usepackage{preamble}
\usepackage{pdfpages}
\begin{document}
\subsection{Model}
\label{sec:Model}
To begin the regression analysis, the wage equation will be estimated using an expanded Mincer equation \citep{MincerA}. Traditional explanatory variables have been used, along with the following variables of interest: disability (binary), part-time (binary), industry (1-digit SIC) and occupation (2-digit SOC), region of UK, public sector employment (binary), under 25 employees (binary) regional unemployment rate. The regional unemployment rate has been included due to its potential impact on wages \citep{Bell}. There are two reasons for this. Firstly, regional unemployment rates may have affect the ability of workers to negotiate higher wages and, secondly, it will help to mitigate the effects of any missing variables which are not time-invariant. \textcolor{red}{Existing literature has explored the theoretical and econometric impact of public sector employment and the firm size (\citet{Longhi} and \citet{Longhi3}), and these are both statistically significant in the model.} There are also theoretical reasons why discrimination may vary by firm size and if it is in the public or private sector. Public firms may be vectors of institutional discrimination, whereas large firms (e.g. FTSE 100 listed) may be under greater pressure to not discriminate, and publish pay gaps. %add more here

Due to the presence of heteroscedasticity in a linear-linear model, the log of hourly wages is preferred. Whilst taking the natural logarithm of the dependent variable  significantly reduces the test statistic, a Breusch-Pagan test for heteroscedasticity gives a p-value of 0.000. Consequently, heteroscedasticity-robust standard errors are used throughout. \textcolor{red}{Autocorrelation or other tests??}

\subsection{Blinder-Oaxaca Decompositions}
\label{sec:B_O_Decomp}
There are two reasons why wages may differ in the labour market. On the one hand, different individuals have different characteristics, such as experience, education and other wage-determining factors. These are what the literature terms \enquote{endowments} \citep{Blinder}. If we can successfully control for every factor relevant to determining one's wage, any remaining gap can be attributed to \enquote{unexplained} factors, which are often termed \enquote{coefficients}. This refers to the coefficients of the multiple linear regression model, which we assume differs between the two groups (e.g. White and Black). %also includes the interactions, depending on what the weighting matrix is??

To analyse the level of the pay gap which can be attributed to \enquote{explained} and \enquote{unexplained} factors, we need to take into account the different endowments possessed by the two groups. In a simple linear regression model, the wage gap (D) can be expressed as \citep{WB}:
\begin{equation}
  D = [\bar{x_{w}} - \bar{x_{b}}]\beta_{b} + [\beta_{w} - \beta_{b}]\bar{x_{b}}
  \label{eq:1}
\end{equation}
We can attribute the first portion to the changes in the endowments (the \enquote{explained} part) and the second portion to the \enquote{unexplained} part (the coefficients and the interaction with endowments):
\begin{equation}
  D = [Explained] + [Unexplained]
  \label{eq:2}
\end{equation}

\section{Decomposition Results}
\label{sec:Decomposition Results}
The results of the multiple linear regression version of the two-fold decomposition in Equation \label{eq:1} can be computed using Ben Jann's Stata Oaxaca-st0001 package \citep{Jann}. The full results are included in \hyperref[sec:appendixA]{Appendix A} in Tables \ref{tab:oaxaca_male} and \ref{tab:oaxaca_female}, with a summary of the output included below.

Both tables begin with the average log hourly wage for the two comparison groups. The pay gap between the two groups can be extracted intuitively, with negative numbers implying a negative contribution to earnings. \enquote*{Non-Whites} is a combined analysis of all ethnic groups, with the other decompositions providing a breakdown of the gap between White workers and the four other ethnic groupings.

\subsection{Males}
\label{sec:Males} %talk about the actual pay gap, in £ or % terms?
The first decomposition (non-White) in Table \ref{tab:oaxaca_male_summary} has a raw gap in log hourly wages of -0.0983, meaning there is a statistically significant wage gap favouring Whites, among men. However, it is important to consider how much of this gap is present when we make like-for-like comparisons. Table \ref{tab:oaxaca_pct_male} shows a quarter of this gap can be \enquote{explained} by differences in characteristics (endowments), with a significant proportion (77.1\%) left \enquote{unexplained} by the model. This can be thought of as discrimination, but only if the regression model fully captures all wage-determining characteristics. It was noted in Section \ref{sec:Setup} that this caveat is more applicable to women than men, however, any factors which disproportionately cause minority ethnic men to earn a lower wage relative to White men will be part of this \enquote{unexplained} component.

\setstretch{1.25}
\import{tables/}{oaxaca_male_summary}
\setstretch{2}

We can specifically isolate \enquote{explained} and \enquote{unexplained} portions of the wage gap for each ethnic grouping. Table \ref{tab:oaxaca_male_summary} shows that the wage gap for Black and Asian male is statistically significant and negative. These data are reported in percentages in Table \ref{tab:oaxaca_pct_male}, with 75.5\% of the Black wage gap \enquote{unexplained} by characteristics, and slightly more for Asian minorities. A large part of this is likely to be direct pay discrimination, although some may be due to issues discussed in Section \ref{sec:limitations}.

\setstretch{1.25}
\import{tables/}{oaxaca_pct_male}
\setstretch{2}

We can see that Chinese and mixed race wages are not statistically different from the wages of White workers. Chinese males have characteristics that are associated with higher wages, but this is suppressed by \enquote{unexplained} factors. Whilst Figure \ref{tab:oaxaca_pct_male} has a large percentage, the level of pay discrimination is likely to be minimal because the raw pay gap is so small (statistically insignificant). The relative size of the unexplained component will be shown shortly in Figure \ref{fig:explained_male_stacked}. For mixed race males, both the raw pay gap and the \enquote{explained} and \enquote*{explained} components are not statistically different from zero.

\subsection{Females}
\label{sec:Females}
The results are very different for female workers. Turning to Table \ref{tab:oaxaca_female_summary}, we can see that non-White females earn more than White females (with a 0.0571 log hourly wage gap). The largest wage gap is for Chinese females at 0.105, with women in each ethnicity earning more than White women. Table \ref{tab:oaxaca_pct_female} shows that, for Black and Asian women, the positive wage gap is despite some \enquote{unexplained} downward pressure on their wage, with their endowments commanding an even larger wage gap than their White counterparts. Asian women, in particular, have strong evidence of pay discrimination, much more than could be due to factors explored in Section \ref{sec:limitations}. In contrast, none of the wage gap for Chinese or mixed race females is due to pay discrimination, with \enquote{unexplained} components being statistically insignificant at the 5\% significance level. Note that whilst ethnic minority females earn more than White females, all ethnicities are subject to the gender wage gap (Section \ref{sec:Literature}).

\setstretch{1.25}
\import{tables/}{oaxaca_female_summary}

\import{tables/}{oaxaca_pct_female}
\setstretch{2}

It has been shown that labour market outcomes differ significantly for males and females. Black and Asian males earn significantly less than White males, with most of this attributed to pay discrimination. In contrast, ethnic minority women typically earn more than their White counterparts, with evidence of pay discrimination only for Black and Asian women. However, this is not to say that there is no gender pay discrimination affecting women of all ethnic groupings, as discussed in Section \ref{sec:Wages}. This is consistent with the literature that ethnic wage gaps for females are much lower, if not reversed, than for males (see \citet{ONSe}, \citet{Metcalf}, and \citet{Longhi3} for UK and \citet{Bayard} for the US). Black and Asian workers of both genders see evidence of pay discrimination, with some pay discrimination affecting Chinese males. %move some of this to the conclusion??

\section{Explained Differences}
\label{sec:explained_diff}
\subsection{Males}
\label{sec:male_explained}
Given that wage differentials can largely be explained, especially for females, discussing the factors contributing to these is beneficial. Graphing the detailed results in Tables \ref{tab:oaxaca_male} and \ref{tab:oaxaca_female} from \hyperref[sec:appendixA]{Appendix A} gives Figures \ref{fig:explained_male_stacked} and \ref{fig:explained_female_stacked}, respectively. Whilst any \enquote{unexplained} components give some evidence of outright pay discrimination, this section on \enquote{explained} components includes a caveat that any \enquote{explained} parts of the wage gap can also be vectors of discrimination. For example, firms may unreasonably refuse to honour equivalent qualifications achieved abroad. Additionally, occupational clustering may occur because minorities perceive an industry to be particularly racist. 

\begin{figure}[h]
\centering
    \title{Explained Portion of Male Wage Differential}
    \import{graphs/}{explained_male_stacked}
    \caption{The portion of the log wage gap which is attributed to various factors, for males, by ethnic group}
    \label{fig:explained_male_stacked}
\end{figure}

It was detailed above that Black and Asian males typically receive a lower wage in part due to differences in endowments. Figure \ref{fig:explained_male_stacked} shows this is mainly due to them working in lower paid occupations/industries (particularly for Black males), along with the fact that they have been working in the UK for less than their White counterparts. For Chinese and mixed race males, their wage is the same as White males. Wages are buoyed down by less labour market experience and tenure with their current firm; this could be due to an increased tendency to switch jobs more frequently compared to White males. Interestingly, this is all despite higher levels of education and working regions associated with higher pay, which combine to make the overall wage gap negligible. Across all ethnicities, males had higher levels of education than their White counterparts, and occupational clustering was important in some cases. \hyperref[sec:appendixB]{Appendix B} shows that ethnic minority males are more likely to work in education, health, and finance than White males, and less likely to work in agriculture, energy, and manufacturing. However Black and Asian minorities are less likely to be managers or work in skilled trades, but instead typically working in care, leisure, and elementary occupations.

As noted previously, these \enquote{explained} components can themselves be a result of (Statistical) discrimination, as detailed in Section \ref{sec:Stat_discrim and signalling}. Occupational clustering could be due to a belief that higher-paying occupations are more \enquote{more racist} or more likely to discriminate; lower levels of experience/tenure in mixed race workers could be due to workers quitting due to an experience of discrimination, for example. Further, fewer years in the UK could mean that English language fluency is poorer, but it could affect wages through discrimination if the individual lacks a \enquote{British accent}. Whilst this is not likely to affect all workers, it could mean that labour market discrimination is present, on top of the direct pay discrimination experienced by Black and Asian males.

\subsection{Females}
\label{sec:females_explained}
For females, we have already noted that the picture is \hyperref[sec:results]{starkly different}. Chinese, Asian and mixed race women tend to have higher wages because they work in higher-paying occupations (hotels/restaurants, education and health - see \hyperref[sec:appendixB]{Appendix B}) than White women, and have higher levels of education. Black women, on the other hand, tend to have higher wages because they work in regions that are associated with higher pay (which itself could be a product of structural discrimination causing regional clustering). Although, this positive wage differential is attenuated somewhat by having varying numbers of years in the UK, consistent with the literature regarding differing labour market outcomes for native and non-native workers (Section \ref{sec:Setup}). Note that differences in marital status and the number of children do not explain any of the wage gap, either to inflate or reduce it. This is increases the reliability of the female analysis, as differing labour market histories regarding maternal/paternal leave do not seem to be affecting the pay of different ethnicities in a significant way. Thus the concerns of Section \ref{sec:Labour Market Outcomes} regarding the bias of the \enquote{unexplained} portion of the wage gap are likely to be minimal, if anything (especially because the \enquote{unexplained} component is not statistically significant for most ethnic groups considered).

\begin{figure}[]
\centering
    \title{Explained Portion of Female Wage Differential}
    \import{graphs/}{explained_female_stacked}
    \caption{The portion of the log wage gap which is attributed to various factors, for females, by ethnic group}
    \label{fig:explained_female_stacked}
\end{figure}

\section{Trends}
\label{sec:time_trends}
The literature surrounding ethnic pay gaps rarely explore the trends. The large sample considered in this paper therefore allows for a detailed analysis of how pay gaps have changed over time. Raw pay gaps were already discussed in Section \ref{sec:Wages}, but it would be beneficial to explore how much of this has historically been due to \enquote{unexplained} factors, which is likely to be due to pay discrimination, at least in part. Blinder-Oaxaca decompositions have been conducted for each year of the sample, where possible, by ethnicity and gender. In some, cases this was not possible, in part due to small sample sizes.

Figure \ref{fig:unexplained_male_line} shows how, for Black and Asian males, the \enquote{unexplained} component has been relatively stationary. A slight deterioration appeared to take place across all three ethnicities from 2008 to 2012, with an improvement to 2019, indicating discrimination may be a significant factor. However, due to this being a highly granular analysis, it is difficult to say with certainty. Mixed race workers seem to have experienced an erosion in the positive \enquote{unexplained} component of the wage gap from 2016 to 2019.

\begin{figure}[h]
\centering
    \title{Male Unexplained Wage Differential}
    \import{graphs/}{unexplained_male_line}
    \caption{The portion of the male wage gap which is unexplained, over time}
    \label{fig:unexplained_male_line}
\end{figure}

When the \enquote{unexplained} female wage gap is considered, a different picture emerges. Figure \ref{fig:unexplained_female_line} is consistent with the findings in Section \ref{sec:Females}, with only Asian females experiencing a consistent negative unexplained wage gap. Similar to males, all ethnic groupings seemed to suffer in the wake of the financial crisis, from 2008 to 2011, which has somewhat recovered since. This indicates the ethnic minorities faced increased discrimination as the economy suffered. Additionally, mixed race females also experienced an increasingly negative \enquote{unexplained} wage gap from 2016 to 2019. This may indicate increased discrimination, but may also take into account factors that have affected mixed race workers more than other ethnicities (which seems relatively unlikely).

\begin{figure}[h]
\centering
    \title{Female Unexplained Wage Differential}
    \import{graphs/}{unexplained_female_line}
    \caption{The portion of the female wage gap which is unexplained, over time}
    \label{fig:unexplained_female_line}
\end{figure}

\section{Conclusions}
\label{sec:conclusions}



\textcolor{red}{FEs - why we can't do fixed effects due to many time-invariant variables, leadng to significant bias \citep{Heitmueller}. Convert log pay gaps into £ ones. Unexplained gaps also incorporate unobserved perceived benefits of working in a particular low-paid job.
Mention how part-time work explains some of the gap - and link this to the summary stats at the start about how more likely to be working part-time. Same with married too.
Unexplained component is an upper bound for discrimination, as can include any unobservables.Although not if industry/occupation is endogenous
Same or different conclusions to \citep{Longhi2} - see the exec summary, as they had men/women of all groups being paid less.
Also compare findings to \citep{Shields} and \citep{ONSe}. Compare the coeffs for each regression, perhaps Whites have a higher return to potential education? This could cause the endogeniety of education, and make it rational for minorities to invest in education less.}


\ifstandalone
\bibliography{essaybib}
\fi
\end{document}