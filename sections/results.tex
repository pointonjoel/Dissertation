\documentclass[class=article, crop=false]{standalone}
\usepackage[subpreambles=true]{standalone}
\usepackage{import}
\usepackage{preamble}
\usepackage{pdfpages}
\begin{document}
\section{Methodology}
\label{sec:Methodology}
\subsection{Model}
\label{sec:Model}
To begin the decomposition analysis, a wage equation will be estimated using an expanded Mincer equation \citep{MincerA}. Traditional explanatory variables have been used, along with the following variables of interest: disability (binary), part-time (binary), industry (1-digit SIC) and occupation (2-digit SOC), region of \acrshort{uk}, public sector employment (binary), under 25 employees (binary) regional unemployment rate. The regional unemployment rate has been included as it may affect the ability of workers to negotiate higher wages \citep{Bell} and, secondly, it will help to mitigate the effects of any missing variables which are not time-invariant. Existing literature has explored the theoretical and econometric impact of public sector employment and the firm size (\citet{Longhi} and \citet{Longhi3}); public firms may be vectors of institutional discrimination, whereas large firms (e.g. FTSE 100 listed) may be under greater pressure to publish pay gaps and not discriminate. %add more here

Due to the presence of heteroscedasticity in a linear-linear model, the log of hourly wages is preferred. Whilst taking the natural logarithm of the dependent variable significantly reduces the test statistic, a Breusch-Pagan test for heteroscedasticity gives a p-value of 0.000. Consequently, heteroscedasticity-robust standard errors are used throughout.

\subsection{Blinder-Oaxaca Decompositions}
\label{sec:B_O_Decomp}
There are two reasons why wages may differ in the labour market. On the one hand, different individuals have different characteristics, such as experience, education and other wage-determining factors. These are what the literature terms \enquote{endowments} \citep{Blinder}. If one can successfully control for every factor relevant to determining wages, any remaining gap can be attributed to \enquote{unexplained} factors, which are often termed \enquote{coefficients}. This refers to the coefficients of the multiple linear regression model, which we assume differs between the two groups (e.g. White and Black). %also includes the interactions, depending on what the weighting matrix is??

To analyse the level of the pay gap which can be attributed to \enquote{explained} and \enquote{unexplained} factors, we need to take into account the different endowments possessed by the two groups. In a simple linear regression model, the wage gap (D) can be expressed as \citep{WB}:
\begin{equation}
  D = [\bar{x_{w}} - \bar{x_{b}}]\beta_{b} + [\beta_{w} - \beta_{b}]\bar{x_{b}}
  \label{eq:1}
\end{equation}
We can attribute the first portion to the changes in the endowments (the \enquote{explained} part) and the second portion to the \enquote{unexplained} part (the coefficients and the interaction with endowments):
\begin{equation}
  D = [Explained] + [Unexplained]
  \label{eq:2}
\end{equation}

\section{Decomposition Results}
\label{sec:Decomposition Results}
The results of the multiple linear regression version of the two-fold decomposition in Equation \ref{eq:1} can be computed using Ben Jann's Stata Oaxaca-st0001 package \citep{Jann}. The full results are included in \hyperref[sec:appendixA]{Appendix A} in Tables \ref{tab:oaxaca_male} and \ref{tab:oaxaca_female}, with a summary of the output included below.

Both tables show the mean log hourly wage for the two comparison groups. The pay gap between the two groups can be extracted intuitively, with negative numbers implying a negative contribution to earnings. \enquote{Non-Whites} is a combined analysis of all ethnic groups, with the other decompositions providing a breakdown of the gap between White workers and the four other ethnic groupings.

\subsection{Males}
\label{sec:Males} %talk about the actual pay gap, in £ or % terms?
The first decomposition (non-White) in Table \ref{tab:oaxaca_male_summary} has a raw gap in log hourly wages of -0.0983, meaning there is a statistically significant wage gap favouring Whites, among men. However, it is important to consider how much of this gap is present when we make like-for-like comparisons. Table \ref{tab:oaxaca_pct_male} shows a quarter of this gap can be \enquote{explained} by differences in characteristics (endowments), with a significant proportion (77.1\%) left \enquote{unexplained} by the model. This can be thought of as discrimination, but only if the regression model fully captures all wage-determining characteristics. It was noted in Section \ref{sec:Setup} that this caveat is more applicable to women than men, however, any factors which disproportionately cause minority ethnic men to earn a lower wage relative to White men will be part of this \enquote{unexplained} component.

\setstretch{1.25}
\import{tables/}{oaxaca_male_summary}
\setstretch{2}

We can isolate the \enquote{explained} and \enquote{unexplained} portions of the wage gap for each ethnic grouping. Table \ref{tab:oaxaca_male_summary} shows that the wage gap for Black and Asian males is statistically significant and negative. Table \ref{tab:oaxaca_pct_male} shows that 75.5\% of the Black wage gap (and slightly more of the Asian wage gap) is \enquote{unexplained} by characteristics. A large part of this is likely to be direct pay discrimination, although some may be due to issues discussed in Section \ref{sec:limitations}.

\setstretch{1.25}
\import{tables/}{oaxaca_pct_male}
\setstretch{2}

We can see that Chinese and mixed-race wages are not statistically different from the wages of White workers. Chinese males have characteristics that are associated with higher wages, but this is suppressed by \enquote{unexplained} factors. Whilst Figure \ref{tab:oaxaca_pct_male} has large percentages for Chinese workers, the level of pay discrimination is likely to be minimal because the raw pay gap is so small (statistically insignificant). The relative size of the unexplained component will be shown shortly in Figure \ref{fig:explained_male_stacked}. For mixed-race males, both the raw pay gap and the \enquote{explained} and \enquote*{explained} components are not statistically different from zero.

\subsection{Females}
\label{sec:Females}
The results are very different for female workers. Turning to Table \ref{tab:oaxaca_female_summary}, we can see that non-White females earn more than White females (with a 0.0571 log hourly wage gap). The largest wage gap is for Chinese females at 0.105, with women of each ethnic minority earning more than White women. Table \ref{tab:oaxaca_pct_female} shows that, for Black and Asian women, the positive wage gap is despite some \enquote{unexplained} downward pressure on their wage, with their endowments commanding an even larger positive wage gap. Asian women, in particular, have strong evidence of pay discrimination, much more than could be due to factors explored in Section \ref{sec:limitations}. In contrast, neither the Chinese nor mixed-race female wage gap is due to pay discrimination, with \enquote{unexplained} components being statistically insignificant at the 5\% significance level. Note that whilst ethnic minority females earn more than White females, all ethnicities are subject to the gender wage gap (see Section \ref{sec:Literature} and Section \ref{sec:Wages}).

\setstretch{1.25}
\import{tables/}{oaxaca_female_summary}

\import{tables/}{oaxaca_pct_female}
\setstretch{2}

%It has been shown that labour market outcomes differ significantly for males and females. Black and Asian males earn significantly less than White males, with most of this attributed to pay discrimination. In contrast, ethnic minority women typically earn more than their White counterparts, with evidence of pay discrimination only for Black and Asian women. However, this is not to say that there is no gender pay discrimination affecting women of all ethnic groupings, as discussed in Section \ref{sec:Wages}. This is consistent with the literature that ethnic wage gaps for females are much lower, if not reversed, than for males (see \citet{ONSe}, \citet{Metcalf}, and \citet{Longhi3} for \acrshort{uk} and \citet{Bayard} for the US). Black and Asian workers of both genders see evidence of pay discrimination, with some pay discrimination affecting Chinese males. %move some of this to the conclusion??

\section{Explained Differences}
\label{sec:explained_diff}
\subsection{Males}
\label{sec:male_explained}
Given that wage differentials can largely be explained, especially for women, discussing the factors contributing to these is beneficial. Whilst any \enquote{unexplained} components give some evidence of outright pay discrimination, this section on \enquote{explained} components includes a caveat that any \enquote{explained} parts of the wage gap can also be vectors of discrimination. For example, occupational clustering may occur because minorities perceive an industry to be particularly racist. 

Graphing the detailed results in Tables \ref{tab:oaxaca_male} and \ref{tab:oaxaca_female} from \hyperref[sec:appendixA]{Appendix A} gives Figures \ref{fig:explained_male_stacked} and \ref{fig:explained_female_stacked}, respectively. These stacked bar charts show any positive contributions to wages above the zero-line, and any negative contributions to wages below. The net effect of these ($\text{positive}-\text{negative}$) is the total shown by the black scatter mark, corresponding to the total pay gaps outlined in Tables \ref{tab:oaxaca_female_summary}. Both the unexplained and the explained contributions to wages, along with their magnitudes and relative sizes, can therefore be easily compared.

\begin{figure}[h]
\centering
    \title{Explained Portion of Male Wage Differential}
    \import{graphs/}{explained_male_stacked}
    \caption{The portion of the log wage gap which is attributed to various factors, for males, by ethnic group}
    \label{fig:explained_male_stacked}
\end{figure}

It was detailed above that Black and Asian males typically receive a lower wage in part due to differences in endowments. Figure \ref{fig:explained_male_stacked} shows this is mainly due to them working in lower-paid occupations/industries, along with the fact that they have been working in the \acrshort{uk} for less than their White counterparts, and experience large unexplained components. For Chinese and mixed-race males, mean wages are not statistically different to the wages of White males. They are buoyed down by less labour market experience and tenure with their current firm; this could be due to an increased tendency to switch jobs more frequently compared to White males. Interestingly, higher levels of education and working regions associated with higher pay combine to make the overall wage gap negligible. Across all ethnicities, males had higher levels of education than their White counterparts, and occupational clustering was important in some cases. \hyperref[sec:appendixB]{Appendix B} shows that ethnic minority males are more likely to work in education, health, and finance than White males, and less likely to work in agriculture, energy, and manufacturing. However, Black and Asian minorities are less likely to be managers or work in skilled trades, and instead typically work in care, leisure, and elementary occupations.

As noted previously, these \enquote{explained} components can themselves be a result of (statistical and structural) discrimination, as detailed in Section \ref{sec:Stat_discrim and signalling}. Occupational clustering could be due to a belief that some higher-paying occupations are more \enquote{more racist} or more likely to discriminate; lower levels of experience/tenure for mixed-race workers could be due to workers quitting due to an experience of discrimination, for example. Further, fewer years in the \acrshort{uk} could mean that English language fluency is poorer, but it could affect wages through discrimination if the individual lacks a \enquote{British accent}. Whilst this is not likely to affect all workers, it could mean that labour market discrimination is present, on top of the direct pay discrimination experienced by Black and Asian males.

\subsection{Females}
\label{sec:females_explained}
We have already noted that the picture is starkly different for women. Chinese, Asian, and mixed-race women tend to have higher wages because they work in higher-paying occupations than White women (hotels/restaurants, education and health --- see \hyperref[sec:appendixB]{Appendix B}), and have higher levels of education. Black women, on the other hand, tend to have higher wages because they work in regions that are associated with higher pay (which itself could be a product of structural discrimination causing regional clustering). Although, this positive wage differential is attenuated somewhat by an unexplained component. All minorities have wages depressed by varying numbers of years in the \acrshort{uk}, consistent with the literature regarding differing labour market outcomes for native and non-native workers (Section \ref{sec:Setup}). Note that differences in marital status and the number of children do not explain a significant portion of the wage gap. This is increases the reliability of the female analysis, as differing labour market histories do not seem to be affecting the pay of different ethnicities in a significant way. Thus the concerns of Section \ref{sec:Labour Market Outcomes} regarding the bias of the \enquote{unexplained} portion of the wage gap are likely to be minimal, if anything (especially because the \enquote{unexplained} component is not statistically significant for most ethnic groups considered).

\begin{figure}[]
\centering
    \title{Explained Portion of Female Wage Differential}
    \import{graphs/}{explained_female_stacked}
    \caption{The portion of the log wage gap which is attributed to various factors, for females, by ethnic group}
    \label{fig:explained_female_stacked}
\end{figure}

\section{Trends}
\label{sec:time_trends}
The literature surrounding ethnicity pay gaps rarely explores the trends. The large sample considered in this paper allows for a detailed analysis of how pay gaps have changed over time. Raw pay gaps were already discussed in Section \ref{sec:Wages}, but it would be beneficial to explore how much of this has historically been due to \enquote{unexplained} factors, which is likely to be due to pay discrimination, at least in part. Blinder-Oaxaca decompositions have been conducted for each year of the sample, where sample size allows, by ethnicity and gender.

Figure \ref{fig:unexplained_male_line} shows that, for Black and Asian males, the \enquote{unexplained} component has been relatively stationary, with some improvement over the sample. A slight deterioration appeared to take place across all three ethnicities from 2008 to 2012, with an improvement through to 2019, indicating discrimination may be a significant factor. However, due to this being a highly granular analysis, it is difficult to say with certainty. Mixed-race workers seem to have experienced an erosion in their largely positive \enquote{unexplained} component of the wage gap from 2016 to 2019.

\begin{figure}[h]
\centering
    \title{Male Unexplained Wage Differential}
    \import{graphs/}{unexplained_male_line}
    \caption{The portion of the male wage gap which is unexplained, over time}
    \label{fig:unexplained_male_line}
\end{figure}

When the \enquote{unexplained} female wage gap is considered, a different picture emerges. Figure \ref{fig:unexplained_female_line} is consistent with the findings in Section \ref{sec:Females}, with only Asian females experiencing a consistent negative unexplained wage gap. Similar to males, all ethnic groupings seemed to experience increasingly negative \enquote{unexplained} wage gaps in the wake of the financial crisis, from 2008 to 2011, with a partial recovery since. This indicates that ethnic minorities faced increased discrimination as the economy suffered. Additionally, mixed-race females also experienced an increasingly negative \enquote{unexplained} wage gap from 2016 to 2019. This may indicate increased discrimination, but may also take into account factors that have affected mixed-race workers more than other ethnicities (which seems relatively unlikely).

\begin{figure}[h]
\centering
    \title{Female Unexplained Wage Differential}
    \import{graphs/}{unexplained_female_line}
    \caption{The portion of the female wage gap which is unexplained, over time}
    \label{fig:unexplained_female_line}
\end{figure}

% FEs - why we can't do fixed effects due to many time-invariant variables, leadng to significant bias \citep{Heitmueller}. Convert log pay gaps into £ ones. Unexplained gaps also incorporate unobserved perceived benefits of working in a particular low-paid job.
% Unexplained component is an upper bound for discrimination, as can include any unobservables.Although not if industry/occupation is endogenous
%\textcolor{red}{Same or different conclusions to \citep{Longhi2} - see the exec summary, as they had men/women of all groups being paid less.}


\ifstandalone
\bibliography{essaybib}
\fi
\end{document}