\documentclass[class=article, crop=false]{standalone}
\usepackage[subpreambles=true]{standalone}
\usepackage{import}
\usepackage{preamble}
\usepackage{pdfpages}
\begin{document}
pre-labour market experiences can have a lasting effect on future earnings (\citet{Altonji}, \citet{Hedman}, and \citet{Berthoud}). 

\section{Literature}
\label{sec:Literature}
%the link here is tenuous/non-existent
In the UK, outcomes for non-White workers are very heterogeneous. Individuals of Bangladeshi ethnicity typically receive the lowest pay, even when controlling for differences in endowments such as education and occupation. In contrast, Chinese, Indian and Mixed Race individuals receive a higher median annual pay than White British employees \citep{ONSe}. The positive pay gap for Chinese workers is often attributed to a relative drive for educations (\citet{Leslie} and \citet{Berthoud}). Outcomes also vary depending on whether the individual is born in the UK or not; the pay gap for Black/African/Caribbean widens from 7.7\% to 15.3\% for those born overseas \citep{ONSe}, potentially due to the unwillingness of employees to recognise the equivalence of foreign qualifications.

Research often finds vastly different outcomes even within ethnic groups \citep{Heath}, with Black African and Caribbean men facing larger pay gaps than Black Mixed workers, with a similar picture for women. Additionally, Pakistani and Bangladeshi men face some of the largest pay gaps, in part due to their concentration in (semi-)routine occupations. Regional variations in pay gaps also exist \citep{ONSe}, with the largest pay gap between White and non-White workers being in London at 21.7\%. Whilst those living in London can expect to earn a higher wage compared to the rest of the population, minorities are at a relative disadvantage. 

The literature often distinguishes \enquote{explained} and \enquote{unexplained} components of the pay gap. Ethnic minorities may receive higher or lower than White British workers for several reasons. Higher (or lower) levels of education can explain wage differentials, with the literature also focussing on the importance of occupational clustering \citep{Brynin}. These are two examples of potentially many components of the \enquote{explained} wage gap. The \enquote*{unexplained} portion includes pay discrimination \citep{Blackaby}, but also any aspects of pay that are unexplained by the model and vary by ethnicity.

The distinction between the \enquote{explained} and \enquote{unexplained} components implies a subtle distinction between direct labour market discrimination and previous discrimination faced by parents and grandparents \citep{Lundberg}. The former is included in the \enquote{unexplained} wage gap. However, historical discrimination can affect education and occupation decisions, as minorities that believe they will experience discrimination in a particular occupation will be more likely to avoid it. Therefore, it is included within the \enquote{unexplained} component. The literature points to the pay gap comprising of a mix of unexplained components and differences in characteristics. However, the majority of research is consistent with a not-insignificant level of discrimination \citep{Metcalf}, particularly for Pakistani but also Black employees, at 58\% and 21\% unexplained, respectively \cite[p.~100]{Blackaby}.

Such techniques focus on the pay gap at the average wage. However, ethnic pay gaps vary depending on the position in the wage distribution. \citet{BoE} find that over the last 25 years ethnic pay gaps have fallen by 50\% over the whole distribution, and attribute the effect at the lower end to minimum wage policy. \citet{Gove} also find that the pay gap has been falling at the lower end of the distribution due to the minimum wage. However, there has been little research into the trend over time. Whilst ethnic pay gaps increased and then fell towards the end of the 20th century (CITE) there has been little evidence since (\citet{Metcalf} and \citet{Blackaby}), particularly for women.

Outcomes also vary by gender, where male ethnic minorities have a relative disadvantage, whereas female ethnic minorities seem to outperform White women \citep{Metcalf}, although all women are subject to the gender pay gap \citep{Heath}. Whilst the motherhood wage penalty and gender pay gaps have been decreasing \citet{BoE}, particularly for high earners, it is still significant, largely due to labour market histories differing between men and women \citep{Olsenb}. \cite[p.~216]{Waldfogel} found that the penalty was mostly due to labour market absence causing a relative reduction in experience, combined with a tendency to enter part-time work, and potentially due to \enquote*{occupational downgrading}. In contrast to this, the fatherhood premium has been rising from 1980 to 2010, although with significant variation across the earnings distribution \citep{Glauber} and \citep{Cooke}. It is therefore important to control for the marital status and number of children of the workers sampled in this paper.

Ethnic differences are often interlinked with religious beliefs. Hindus and Sikhs face varying pay gaps, with the latter more likely to face lower pay, and outcomes for Muslims are negative but heavily depend on their ethnicity (with Pakistani Muslims facing the worst outcomes) (\citet{Johnston} and \citet{Longhi}). \citep{Metcalf} also found that Muslims face a negative pay gap, with Jews facing a positive one. However, some papers neglect this aspect, citing reduced sample sizes \citep{Brynin}.

Ethnic disadvantages extend beyond pay itself, with many minorities more likely to be unemployed \citep{Heath}. Additionally, the unemployment rates of ethnic minorities tend to increase faster in a recession than White British individuals \citep{Jones}. Consequently, even if pay gaps were to be nonexistent, outcomes would not be equitable. Unemployment may also further worsen minorities' integration into the labour market and wider society. Overall, ethnic inequalities are manifested in a variety of ways, and even though this paper focuses on pay gaps, other labour market disparities exist. Even just within pay gaps, dynamics are nuanced and attributed to a wide range of factors. This paper contributes to the literature by providing a detailed breakdown of the ethnic pay gap across time, not just for men, but also women.

\textcolor{red}{Does this lit review say what I want it to, overall??}

\subsection{Policy}
\label{sec:Policy} %bit of a jump to this part - tenuous link
Well designed policies to curtail pay gaps can lead to a Pareto Improvement, bring both equity and efficiency gains \citep{LundbergB}, along with significant gains to GDP \citep{GOVg}. Some Economists assert that whilst pay reporting is a \enquote*{starting point} and not an \enquote*{end-point}, it is a catalyst for further action and improvements in closing the pay gap \citep{BoE}. However, research has shown how many such initiatives, along with Equal Opportunities Policies, often have \enquote*{little or no impact} \cite[p.~113]{Noon}. Additionally, with the vast majority of research pointing to differences in endowments among workers, pay reporting policies place excessive emphasis on the prejudice of companies rather than addressing past policy failures. Whilst occupational segregation is a complex and endogenously-driven phenomenon, in part due to companies employing minorities in lower occupations (Section \ref{sec:Human Capital Models}), it requires specific policy interventions to ensure (educational and other) outcomes are increasingly equal. The nature of the policy recommendations depends on whether the pay gaps are due to pay discrimination or specific differences in endowments. \textcolor{red}{Same as the conclusion info.}

\ifstandalone
\bibliography{essaybib}
\fi
\end{document}