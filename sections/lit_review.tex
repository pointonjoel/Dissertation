\documentclass[class=article, crop=false]{standalone}
\usepackage[subpreambles=true]{standalone}
\usepackage{import}
\usepackage{preamble}
\usepackage{pdfpages}
\begin{document}
\section{Literature}
\label{sec:Literature}
In light of the above discussion, the literature often distinguishes between pay discrimination and occupational segregation \citep{Brynin}. Additionally, current labour market discrimination (the unexplained part) is separated from previous discrimination faced by parents and grandparents \citep{Lundberg}. They can both affect labour market outcomes, leading to endogeniety, as minorities that believe they will experience discrimination in a particular occupation will be more likely to avoid it.
%the link here is tenuous/non-existent
In the UK the pay gap varies significantly by ethnicity, with Chinese, Indian and Mixed Race individuals receiving a higher median annual pay than White British employees \citep{ONSe}. Individuals of Bangladeshi ethnicity receive the lowest pay, even when controlling for differences in endowments such as education and occupation. This even varies by region, with the largest pay gap being in London at 21.7\%. It also shows that the pay gap is lower for younger than for older employees.

\citep{Berthoud} found that the pay gap for Black/African/Caribbean 7.7\% for British-born workers, but widens to 15.3\% for those born overseas, potentially due to the unwillingness of employees to recognise the equivalence of foreign qualifications. The positive pay gap for Chinese workers is often attributed to a relative drive for educations \citep{Leslie}. However, once personal characteristics/edownments have been taken into account, all ethnicities face lower pay than Whites \citep{Metcalf}. \textcolor{red}{Compare this to my own data (tables showing education levels by ethnicity - maybe a stacked bar would be better, similar to the employment one!), as the above study suggests that African (and Indians) has more education.}

Once the characteristics of workers are taken into account, research often finds vastly different outcomes \citep{Heath}. There is a large amount of heterogeneity within Black workers, with Black African and Caribbean men facing a larger pay gaps than Black Mixed workers, with a similar picture for women. Additionally, Pakistani and Bangladeshi men face some of the largest pay gaps, in part due to their concentration in (semi-)routine occupations. However, \citep{Heath} notes that the pay gaps for women may in part due to equally poor outcomes for White women. It is also important to note how Indian workers have managed to reduce the pay gap over time, yet credible reasons as to why this is the case are lacking.

Much of academic research points to the pay gap being a mix of unexplained components and differences in characteristics. Researchers using decomposition analyses are often attribute the unexplained component to a mix of discrimination, omited variables and other complicating factors \citep{Blackaby}. However, the majority of research is consistent with a not-insignificant level of discrimination \citep{Metcalf}, particularly for Pakistani but also Black employees, at 58\% and 21\% unexplained, respectively \cite[p.~100]{Blackaby}.

The motherhood wage penalty also receives a significant amount of attention in the literature due to the significant effect of having children on wages  - I need a reference as \citep{BoE} and \citep{Nizalova} don't exactly cover this. Whilst it has been decreasing, particularly for high earners, it is still significant. \cite[p.~216]{Waldfogel} found that the penalty was largely due to the time out of the labour market reducing experience, the tendency to enter part time work, and potentially due to \enquote*{occupational downgrading}. In contrast to this, the fatherhood premium has been rising from 1980 to 2010, with significant variation across the earnings distribution \citep{Glauber} and \citep{Cooke}. Pay gaps also differ for males and females, with males facing the greatest pay gaps \citep{Metcalf}.

Additionally, ethnic pay gaps vary depending on the position in the wage distribution. \citep{BoE} find that over the last 25 years ethnic pay gaps have fallen by 50\% over the whole distribution, and attribute the effect at the lower end to minimum wage policy. There is little evidence to point to the trend over time. Whilst ethnic pay gaps increased and then fell towards the end of the 20th century (CITE) there has been little evidence since (\citet{Metcalf} and \citet{Blackaby}), particularly for women. However, there has been evidence of a falling pay gap at the lower end of the distribution due to the minimum wage \citep{Gove}.

Ethnic differences are often interlinked with religious beliefs. Hindus and Sikhs face varying pay gaps, with the latter more likely to face lower pay, and outcomes for Muslims are negative but heavily depend on their ethnicity (with Pakistani Muslims facing the worst outcomes) (\citet{Johnston} and \citet{Longhi}). \citep{Metcalf} also found that Muslims face a negative pay gap, with Jews facing a positive one. However, some papers neglect this aspect, citing reduced sample sizes \citep{Brynin}.

Ethnic disadvantages extend beyond pay itself, with many minorities more likely to be unemployed \citep{Heath}. Additionally, unemployment rates of ethnic minorities tend to increase faster in a recession than White British individuals \citep{Jones}. Consequently, even if pay gaps were to be nonexistent, outcomes would not be equitable. Unemployment may also further worsen minorities' integration into the labour market and wider society.

\textcolor{red}{Does this lit review say what I want it to, overall??}

\subsection{Policy}
\label{sec:Policy}
Well designed policies to curtail pay gaps can lead to a Pareto Improvement, bring both equity and efficiency gains \citep{LundbergB}, along with significant gains to GDP \citep{BoE}. Some Economists assert that whilst pay reporting is a \enquote*{starting point} and not an \enquote*{end-point}, it is a catalyst for further action and improvements in closing the pay gap \citep{BoE}. However, research has shown how many such initiatives, along with Equal Opportunities Policies, often have \enquote*{little or no impact} \cite[p.~113]{Noon}. Additionally, with the vast majority of research pointing to differences in endowments among workers, such policies imply that the pay gap lies with companies' prejudice rather than the past policy failures. Whilst occupational segregation is a complex and endogenously-driven phenomenon, in part due to companies employing minorities in lower occupations (Section \ref{sec:Human Capital Models}), and requires specific policy intervention to ensure (educational and other) outcomes are increasingly equal. The nature of the policy recommendations depend on whether the pay gaps are due to pay discrimination or specific differences in endowments. \textcolor{red}{Same as the conclusion info.}

\ifstandalone
\bibliography{essaybib}
\fi
\end{document}