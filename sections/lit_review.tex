\documentclass[class=article, crop=false]{standalone}
\usepackage[subpreambles=true]{standalone}
\usepackage{import}
\usepackage{preamble}
\usepackage{pdfpages}
\begin{document}

\section{Literature}
\label{sec:Literature}
%the link here is tenuous/non-existent
In the \acrshort{uk}, outcomes for non-White workers are very heterogeneous. Individuals of Bangladeshi ethnicity typically receive the lowest pay, with Chinese, Indian, and mixed-race workers receiving a higher median annual pay than White British employees \citep{ONSe}. Research often finds vastly different outcomes even within ethnic groups \citep{Heath}. Black African and Caribbean men face larger pay gaps than Black mixed-race workers, with a similar picture for women. Additionally, Pakistani and Bangladeshi men have some of the largest pay gaps. %in part due to their concentration in (semi-)routine occupations. 

Many techniques focus on the pay gap at the average wage. However, ethnicity pay gaps vary depending on the position in the wage distribution, and across time. \citet{BoE} find that over the last 25 years ethnicity pay gaps have fallen by 50\% over the whole distribution, and attribute the effect at the lower end to minimum wage policy. \citet{Gove} also find that the pay gap has been falling at the lower end of the distribution due to the minimum wage. However, there has been little research into the trend over time. Whilst raw (unconditional) ethnicity pay gaps have fallen from 30\% to 20\% from the mid-1990s to 2018 \cite[p.~11]{BoE}, there has been little evidence of a fall towards the end of the sample, particularly for women (\citet{Metcalf} and \citet{Blackaby}).

\subsection{Gender}
\label{sec:Gender}
Outcomes also vary by gender, albeit with women of every ethnicity subject to the gender pay gap. \citet{Longhi3} found that the ethnicity pay gap is larger for men than for women, such that ethnic minority women have a smaller pay gap (relative to White women) than for ethnic minority men (relative to White men). However, \citet{Metcalf} even found that female ethnic minorities seem to outperform White women, although all women are subject to the gender pay gap \citep{Heath}. This paper will therefore separate male and female analysis, to examine the level and determinants of the ethnic wage gap for both genders.

%Whilst the motherhood wage penalty and gender pay gaps have been decreasing \citet{BoE}, particularly for high earners, it is still significant, largely due to labour market histories differing between men and women \citep{Olsenb}. \cite[p.~216]{Waldfogel} found that the penalty was mostly due to labour market absence causing a relative reduction in experience, combined with a tendency to enter part-time work, and potentially due to \enquote*{occupational downgrading}. In contrast to this, the fatherhood premium has been rising from 1980 to 2010, although with significant variation across the earnings distribution \citep{Glauber} and \citep{Cooke}. This paper will therefore control for the marital status and number of children of workers. Whilst \acr

\subsection{Components of the Wage Gap}
\label{sec:components}
There are many reasons why ethnic minorities are paid different wages to White workers and. as such, the literature often distinguishes \enquote{explained} and \enquote{unexplained} components of the pay gap. The \enquote{explained} component refers to the differences in characteristics (education, experience, marital status, etc.), whereas the \enquote{unexplained} portion is any part of the wage gap which cannot be explained by the model used. It is important to note that the \enquote{explained} wage gap can encompass an element of endogenous discrimination. Additionally, the \enquote{unexplained} portion is not necessarily entirely discrimination, especially if the model is not perfect; the level of discrimination is difficult to measure precisely. Instead it can be seen as an upper-bound on the level of direct pay discrimination. The literature points to the pay gap comprising of a mix of unexplained components and differences in characteristics. However, the majority of research is consistent with a not-insignificant level of discrimination \citep{Metcalf}, particularly for Black employees, with 49\% of the earnings gap left unexplained (calculated from \citet[p.~374]{Blackaby}).

The distinction between the \enquote{explained} and \enquote{unexplained} components implies a subtle distinction between direct labour market discrimination and previous discrimination faced by parents and grandparents \citep{Lundberg}. The former is included in the \enquote{unexplained} wage gap. However, historical discrimination can affect education and occupation decisions, as minorities that believe they will experience discrimination in a particular occupation will be more likely to avoid it. Therefore, it is included within the \enquote{unexplained} component. These pre-labour market experiences can have a lasting effect on future earnings (\citet{Altonji}, \citet{Hedman}, and \citet{Berthoud}).

\subsection{The Explained Wage Gap}
Ethnic minorities may receive higher or lower than White British workers for several reasons. The wage equation proposed by \citet{MincerA} outlines how education and experience are key determinants of wages. The positive pay gap for Chinese workers is often attributed to a relative drive for educations (\citet{Leslie} and \citet{Berthoud}), and all minorities tend to have higher levels of education, levelling wages \citep{Longhi2}. The literature also focuses on the importance of occupational clustering, where, over time, minorities become over-represented in certain occupations and under-represented in others. Whilst clustering often negatively impact wages, Chinese and Indian workers earn a wage premium thanks to occupational clustering \citep{Brynin}. As noted in Sections \ref{sec:Stat_discrim and signalling} and \ref{sec:components}, occupational clustering can be indicative of statistical and structural racism, despite being encompassed within the \enquote{explained} portion of the wage gap.

Aside from traditional determinants of wages, several variables have been used in the literature. Ethnic wage gaps vary depending on whether the individual is born in the \acrshort{uk} or not (\citet{Brynin}, \citet{ONSe} and \citet{Shields}); the pay gap for Black/African/Caribbean widens from 7.7\% to 15.3\% for those born overseas \citep{ONSe}, potentially due to the unwillingness of employees to recognise the equivalence of foreign qualifications. Even the region within the \acrshort{uk} seems to be an important factor, with the largest pay gap between White and non-White workers being in London at 21.7\% \citep{ONSe}. Therefore, complex factors are at play; those working in London can expect to earn a higher wage compared to the rest of the population, but minorities are at a relative disadvantage.

Ethnic differences are often interlinked with religious beliefs. Hindus and Sikhs face varying pay gaps, with the latter more likely to face lower pay, and outcomes for Muslims are negative but heavily depend on their ethnicity (with Pakistani Muslims facing the worst outcomes) (\citet{Johnston} and \citet{Longhi}). \citep{Metcalf} also found that Muslims face a negative pay gap, with Jews facing a positive one. However, some papers neglect this aspect, citing reduced sample sizes \citep{Brynin}.

\subsection{Employment}
\label{sec:Employment}
Ethnic disadvantages extend beyond pay itself, with many minorities more likely to be unemployed \citep{Heath}. Additionally, the unemployment rates of ethnic minorities tend to increase faster in a recession than White British individuals \citep{Jones}. Consequently, even if pay gaps were to be nonexistent, outcomes would not be equitable. Unemployment may also further worsen minorities' integration into the labour market and wider society. Overall, ethnic inequalities are manifested in a variety of ways, and even though this paper focuses on pay gaps, other labour market disparities exist. Even just within pay gaps, dynamics are nuanced and attributed to a wide range of factors. This paper contributes to the literature by providing a detailed breakdown of the ethnicity pay gap across time, not just for men, but also women.

\subsection{Policy}
\label{sec:Policy} %bit of a jump to this part --- tenuous link
Various policies have been implemented to curtail the ethnicity pay gap. Well designed policies bring both equity and efficiency gains: a Pareto Improvement \citep{LundbergB}, along with significant gains to GDP \citep{GOVg}. The largest \acrshort{uk} firms engage in pay reporting, which is seen as a \enquote*{starting point} and not an \enquote*{end-point} \citep{BoE}. However, research has shown how many such initiatives, along with Equal Opportunities Policies, often have \enquote*{little or no impact} \cite[p.~113]{Noon}. Additionally, with the vast majority of research pointing to differences in endowments among workers, pay reporting policies place excessive emphasis on the prejudice of companies rather than addressing past policy failures. Whilst occupational segregation is a complex and endogenously-driven phenomenon, in part due to companies employing minorities in lower occupations (Section \ref{sec:Human Capital Models}), it requires specific policy interventions to ensure (educational and other) outcomes are increasingly equal. The nature of the policy recommendations, therefore, depends on whether the pay gaps are due to pay discrimination or specific differences in endowments.

\ifstandalone
\bibliography{essaybib}
\fi
\end{document}