\documentclass[class=article, crop=false]{standalone}
\usepackage[subpreambles=true]{standalone}
\usepackage{import}
\usepackage{preamble}
\usepackage{pdfpages}
\begin{document}
\section{Introduction}
\label{sec:introduction}
While elements of equality for women have been enshrined in UK law as early as 1882 \citep{GOVb}, equal pay for women was only introduced as late as 1970 \citep{GOVc}, and strengthened in 1975 by the Sex Discrimination Act of 1975 \citep{GOVd}. Racial wage discrimination became illegal in 1965 \citep{GOV} with the Race Relations Act, in light of increased immigration after WWII. The law was later strengthened in 1968 and 1976 \citep{Sooben}. Yet despite criticism of the legislation in the wake of the Brixton riots of 1981 \citep{Solomos}, the inadequacy of labour market protections for ethnic minorities was not properly addressed until 2010 through the Equality Act \citep{Brown}. Recent criticism of economic models by \citet{Spriggs} brought research of discrimination to the fore, citing with the need for a more objective approach to research. With little recent research into ethnic pay gaps, unlike gender pay gaps \citep{Metcalf}, and even less research into how they have changed over time, this paper seeks to address this.

Pay gaps can occur in two main ways. Either employers directly pay minorities a lower wage for the same job, or workers self-select into an (often lower paid) occupation. The former is less likely, with the latter often causing minorities to cluster in lower paid jobs. This is not to say that self selection into occupations is not a form of discrimination. Often this can be a choice by minorities in response to perceived racism in particular sectors, or by historical pay discrimination causing under-investment in skills and education. Consequently, pay gaps emerge and are often persistent, with many underlying factors at play, making policy solutions complex.

In the next section we explore economic models and theories which attempt to characterise aspects of the labour market, along with the literature surrounding ethnic pay gaps in the UK. Chapter \ref{sec:Data} outlines the details of data collection along with descriptive analysis, which provides the backdrop for decompositions and analysis of endownments. The Blinder-Oaxaca decomposition technique is then outlined and implemented, with a detailed discussion of the results. To conclude, the limitations of the analysis along with areas for future research are noted, along with this paper's main findings. Appendices can be found in Chapter \ref{sec:appendices}.

Amanda's meeting notes:
Change for occup and not industry. 
Similar for industry

Possibly higher status jobs are less racist. But probs not. 

Racist attitudes cause barriers to entry. Or like a tax on employment in a certain activity so pushed away.

Link with models of the Labour Market fit in with the results. 

Stratification researchers get annoyed when we model racism. If racism is structural then there's a lot of endogeniety. 

\ifstandalone
\bibliography{essaybib}
\fi
\end{document}