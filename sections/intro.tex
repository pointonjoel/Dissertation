\documentclass[class=article, crop=false]{standalone}
\usepackage[subpreambles=true]{standalone}
\usepackage{import}
\usepackage{preamble}
\usepackage{pdfpages}
\begin{document}
\pagestyle{plain}
\section{Introduction}
\label{sec:introduction}
Racial wage discrimination became illegal in the \acrshort{uk} in 1965 \citep{GOV} with the Race Relations Act, in light of increased immigration after WWII. The law was later strengthened in 1968 and 1976 \citep{Sooben}. Yet despite criticism of the legislation in the wake of the Brixton riots of 1981 \citep{Solomos}, the inadequacy of labour market protections for ethnic minorities was not addressed until 2010 through the Equality Act \citep{Brown}. With little recent research into ethnicity pay gaps, unlike gender pay gaps \citep{Metcalf}, and even less research into how they have changed over time, this paper seeks to address this. Recent criticism of economic models by \citet{Spriggs}, in response to global anti-racism protests, has shown the need for a more objective approach to discrimination research.

Ethnicity pay gaps are just one, among many, expressions of discrimination in modern economies. Pay gaps are where ethnic minorities are paid different amounts (often less) than native workers and can occur in three main ways. Either employees have different characteristics (economically justifying their different wage), employers directly pay minorities a lower wage for the same job, or workers cluster into lower-paying industries. All three can foster an element of discrimination, and are interlinked. If a job applicant has less education or labour market experience, an employer is legally entitled to pay a lower wage. However, minorities may have less education, for example, because they are not willing to train to work in industries that are perceived to be racist. Over time, this leads to occupational clustering, where minorities are over-represented in some industries and under-represented in others. Consequently, pay gaps emerge and are often persistent, with many underlying factors at play, making policy solutions complex --- even though direct pay discrimination is illegal in the \acrshort{uk}.

%The latter is less likely, with the former causing minorities to cluster into often lower-paid jobs. This itself can be due to the avoidance of occupations that are discriminatory, or due to historical pay discrimination causing under-investment in skills and education.

In the next section, we explore economic models and theories which attempt to characterise aspects of the labour market, along with the literature surrounding ethnicity pay gaps in the \acrshort{uk}. Chapter \ref{sec:Data} outlines the details of data collection along with descriptive analysis, which provides the backdrop for decompositions and analysis of endowments. The Blinder-Oaxaca decomposition technique is then outlined and implemented in Chapter \ref{sec:Analysis}, with a detailed discussion of the results. To conclude, Chapter \ref{sec:Limitations} discusses the limitations of the analysis along with areas for future research and finished by concluding with the main findings of this paper. Appendices can be found in Chapter \ref{sec:appendices}.

\begin{comment}
Amanda's meeting notes:
Change for occup and not industry. 
Similar for industry

Possibly higher status jobs are less racist. But probs not. 

Racist attitudes cause barriers to entry. Or like a tax on employment in a certain activity so pushed away.

Link with models of the Labour Market fit in with the results. 

Stratification researchers get annoyed when we model racism. If racism is structural then there's a lot of endogeniety. 
\end{comment}

\ifstandalone
\bibliography{essaybib}
\fi
\end{document}