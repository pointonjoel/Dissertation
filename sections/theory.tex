\documentclass[class=article, crop=false]{standalone}
\usepackage[subpreambles=true]{standalone}
\usepackage{import}
\usepackage{preamble}
\usepackage{pdfpages}
\begin{document}
\section{Theory}
\label{sec:Theory}
There are 3 main factors contributing to discrimination. The first is individual discrimination, where employers undervalue the productive potential of particular workers (be it female or minority ethnic workers). The second is institutional discrimination, where there are incentives to discriminate against specific groups, leading to self-selection into particular firms, or possibly even labour market practices preventing entry of certain groups into particular labour markets. Thirdly, there is systemic (or structural) discrimination, where society prices different groups of people differently.

\subsection{Human Capital Models}
\label{sec:Human Capital Models}
Various theoretical models have been devised over the past few decades which attempt to explain the presence of discrimination and model these mechanisms. Taste models pioneered by \citet{Becker} and \citet{Arrow} assert that employers have a preference for Whites and therefore only employ minorities at a lower wage rate to compensate them for lower utility, known as individual discrimination. With perfectly competitive markets, firms which are not prejudiced will offer a higher wage until prejudiced employers are forced to exit the market. Alternatively, in such a scenario, firms may employ workers in lower occupations.

Discrimination may therefore exist in practice if firms can collude to allow prejudice to exist long-term, especially if institutions permit this collusion \citep{LundbergB}. Alternatively, if customers are prejudiced and value the output of minorities less, this will lead to a material productivity difference, meaning economic incentives exist for firms to offer them lower wages (systemic discrimination).

\subsection{Statistical Discrimination and Signalling Models}
\label{sec:Stat_discrim and signalling}
Even if direct wage discrimination does not exist, if minorities are less likely to be offered higher wage jobs then they can still remain in a self-fulfilling lower-wage equilibrium \citep{Coate}.

https://www.open.edu/openlearn/people-politics-law/politics-policy-people/economics/economics-explains-discrimination-the-labour-market/content-section-5.3.1
Statistical discrimination occurs in the presence of asymmetric information. If employers believe that non-White workers are less productive minorities can be trapped in a low wage equilibrium due to historical factors (such as??). This is a self-fulfilling phenomena in which discrimination can occur even when minorities are not of lower productivity \citep{Arrow} and \citep{LundbergB}. This is linked to the concept of signalling models explored in \citet{Barr}, whereby two Nash equilibria can occur, due to an inability for minorities to convince employers of their equal productivity. This is also known as institutional discrimination.

If minorities therefore expect to earn less in the labour market, the returns to education are lower, and they are incentives to reduce investment in education. This is a self-fulfilling prophecy, as miniorities have lower education than their White counterparts, and the expectations of the prejudiced employers are realised, when they would not have been otherwise. Human capital theory therefore predicts that ethnic minorities (and women) will earn less than White British men because of this endogeniety of investment in education. Over time, increased underinvestment in education, combined with specialisation in other skills, can lead to comparative advantage with specific ethnicities, further entrenching occupational segregation \citep{Altonji}.

Furthermore, differences in preferences can also determine the occupational distribution of workers \citep{Thaler}. For example, if White men's preferences are such that they are willing to work in recycling centers, when other demographics are much less so, then the scarcity of labour supply will command a higher wage \citep{Altonji}. Aggregating these microeconomic factors leads to segregation into various occupations.

The final aspect which may affect the wages of ethnic minorities is sociological factors. Family structures and difficulties in integration in society may be added to by alienation, possibly due to past experiences of discrimination \citep{Berthoud}. This can hamper employment prospects and outcomes in the labour market.

\subsection{Labour Market Outcomes}
\label{sec:Labour Mkt Outcomes}
The above theoretical models purport that outright pay discrimination by individual employers is likely to be rare. However, differences in pre-labour market experiences can have a lasting effect on future earnings (\citet{Altonji}, \citet{Hedman}, and \citet{Berthoud}). The characteristics of individuals are often termed endownments in the literature, meaning that when minorities enter the labour market they have reduced endownments than their White counterparts, through no fault of their own, but merely following economic incentives formed by other economic agents. \textcolor{red}{This links with comparative advantage.}

\section{Wage Determination}
Aside from discrimination, there are multiple factors which affect the wages of workers. Higher education is rewarded by employers due to increased productivity \citep{Becker}, specific occupations and industries may command a higher wage, and women may earn less if they expect to spend less time in the labour market, investing less in education. Further, those who are married may command a higher wage (\textcolor{red}{due to Chicago Rationale}), as those who are more productive are more attractive in the marriage market. Additional complex effects of marriage on the labour market are explored by \citep{Ahituv}. 

Further, experience in the labour market (and the occupation or industry in particular) increases productivity and increases wages. Conversely, periods out of the labour market, through unemployment or maternal/paternal reasons will have a detrimental impact on wages. Skills can depreciate over time, and employers may offer a lower wage upon return. 

Practical considerations can also affect wages; those living in specific regions (such as London) may naturally earn a higher wage to compensate for increased living costs. Increasing consumer prices for the entirety of the postwar period \citep{WorldBank} has also led to wage rises. Additionally, part time workers may be willing to work for a lower wage if it is supplementing alternative forms of income. Finally, those who are disabled may be subject to similar outcomes to ethnic minorities. Additional factors such as motivation and ambition also determine wages, but cannot be measured easily and are also probably endogenous (similar to the discussion in Section \ref{sec:Stat_discrim and signalling}). 

\ifstandalone
\bibliography{essaybib}
\fi
\end{document}