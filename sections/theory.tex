\documentclass[class=article, crop=false]{standalone}
\usepackage[subpreambles=true]{standalone}
\usepackage{import}
\usepackage{preamble}
\usepackage{pdfpages}
\begin{document}
\section{Theory}
\label{sec:Theory}
Economic discrimination can manifest itself in three ways. The first is individual discrimination, where agents directly treat minorities unfavourably --- be it female, minority ethnic, or other forms of minorities. For example, a firm may choose to employ a White worker over a minority worker when they are equal in every other way. The second form is institutional discrimination, where institutions and practices within the economy work against specific groups at a \enquote*{nonconcious level} \cite[p.~1814]{Ian}, even if individuals themselves do not. Thirdly, there is structural discrimination, which refers to the ongoing perpetuation of disparities in opportunity and outcome \citep{Lawrence}. There are several economic models which serve to shed light on the mechanisms at work within these discriminatory practices, which will now be explored.

\subsection{Taste Models}
\label{sec:Human Capital Models}
Taste models pioneered by \citet{Becker} and \citet{Arrow} assert that employers have a preference for White workers. Therefore, they only employ minorities at a lower wage rate to compensate themselves for receiving a lower utility, paying a price to avoid employing minorities. Even if direct wage discrimination does not occur, firms that are prejudiced may only offer minorities jobs in lower-paying occupations. Therefore, ethnic minorities are forced into earning lower wages, either through being paid lower wages for the same job or only being offered lower occupation jobs \citep{Coate}.

With perfectly competitive markets, firms have a profit incentive to offer minorities a \enquote{fair} wage, attracting talented minorities. That is, non-prejudiced firms pay minorities the same wage as their counterparts until prejudiced employers are forced to exit the market. Discrimination may therefore only exist in practice if firms can collude to allow prejudice to exist long-term, especially if institutions permit this collusion \citep{LundbergB}. Alternatively, if customers are prejudiced and value the output of minorities less, this will lead to a material difference in \acrshort{mrp} (productivity), meaning economic incentives exist for firms to offer them lower wages.

\subsection{Statistical Discrimination and Signalling Models}
\label{sec:Stat_discrim and signalling}
Statistical discrimination is different to taste-based discrimination, in that it is not a result of direct prejudice. Instead, it occurs in the presence of asymmetric information. If employers cannot easily determine the productivity of workers, they may have an irrational perception that minorities are less productive than White workers (\citet{Phelps} and \citet{LundbergB}). Consequently, firms will only offer minorities a lower wage, regardless of their true productivity, causing ethnic minorities to be trapped in a low wage equilibrium \citep{Barr}. A key aspect of this model is that minorities are unable to \enquote{signal} to employers their true productivity, and as such the employer is forced to make assumptions. Although, these assumptions can be based upon (discriminatory) stereotypes and biases.

Statistical models of discrimination built upon the work of \citep{Arrow} show that average differences between minorities and Whites can occur endogenously. These extend beyond the work of \citet{Phelps}, which assumes that discrimination occurs as a result of a combination of imperfect information and exogenous differences between Whites and minorities. We can go so far as to say that minorities' outcomes can be a result of endogenous interactions alone: that is, the assumptions made by employers in the presence of asymmetric information is enough to cause minorities to be trapped in a low wage equilibrium. 

\subsection{Entrenchment}
Group differences can become entrenched as minorities make investment decisions regarding their education. If minorities expect to earn a lower wage, their returns to education, and therefore incentives to invest in education, are lower than is economically efficient. This is a self-fulfilling prophecy; incentives cause minorities to have less education than their White counterparts, and the expectations of the prejudiced employers are realised when they would not have been otherwise \citep{Hanming}. This can lead to comparative advantage in certain job types, further entrenching occupational segregation \citep{Altonji}. The theory, therefore, predicts that ethnic minorities will earn less than White British men because of this endogeneity of investment in education. Therefore, statistical discrimination explains why discrimination can persist even in the presence of perfectly competitive markets \citep{LundbergB}.

Occupational clustering and segregation can also be caused by differences in preferences \citep{Thaler}. For example, if the preferences of White men are such that they are willing to work in recycling centres when other ethnicities are much less willing to, the scarcity of labour supply will command a higher wage \citep{Altonji}. Aggregating these micro-economic factors leads to occupational clustering which can become entrenched over time.

%\subsection{Systemic Discrimination}
%\label{sec:systemic_discrim}
%Systemic discrimination is where society as values the contribution of minorities less than for White workers. For example, if the public value the services of a White waiter more than the services of an ethnic minority waiter, then it provides a material difference in the \acrfull{mrp} of the workers. The firm consequently has an economic incentive to pay the minority a lower wage, even if it is based on irrational value judgements by society. These complex factors are difficult to measure and are not within the scope of this paper.

\subsection{Other Causes of Wage Determination}
\label{sec:other_causes_of_discrimination}
Aside from discrimination, there are multiple sociological factors that affect the wages of workers\footnote{Other factors include difficulties in integrating into society, which can lead to alienation, possibly due to past experiences of discrimination \citep{Berthoud}, hampering labour market outcomes.}. Family structures can vary by ethnicity, and women may earn under-invest in education if they expect to spend less time in the labour market. This can lead to the depreciation of skills and productivity, hampering wages. Additionally, certain ethnicities may invest in higher education more than others, increasing productivity and labour market remunerations \citep{Becker}. Further, those who are married may command a higher wage, as productive workers may be more attractive in the marriage market, and can take advantage of the division of labour within the household\footnote{Additional effects of marriage on the labour market are explored by \citep{Ahituv}.} (\citet{Nakosteen} and \citet{Bardasi}).

Practical considerations can also affect wages; those living in specific regions (such as London) may naturally earn a higher wage to compensate for increased living costs. Additionally, part-time workers may be willing to work for a lower wage if it is supplementing alternative forms of income. Economy-wide wages have also increased as a result of increasing consumer prices for the entirety of the postwar period \citep{WorldBank}. All of the above factors will be considered in our model where possible. Additional factors such as motivation and ambition also determine wages, but cannot be measured easily and are also probably endogenous (similar to the discussion in Section \ref{sec:Stat_discrim and signalling}).

\ifstandalone
\bibliography{essaybib}
\fi
\end{document}