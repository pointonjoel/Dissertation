\documentclass[class=article, crop=false]{standalone}
\usepackage[subpreambles=true]{standalone}
\usepackage{import}
\usepackage{preamble}
\usepackage{pdfpages}
\begin{document}
\section{Theory}
\label{sec:Theory}
Discrimination can occur in three ways. The first is individual discrimination, where employers undervalue the productive potential of particular workers (be it female or minority ethnic workers). The second is institutional discrimination, where there are incentives to discriminate against specific groups, leading to self-selection into particular firms, or possibly even labour market practices preventing entry of certain groups into particular labour markets. Thirdly, there is systemic (or structural) discrimination, where society prices different groups of people differently.

\subsection{Human Capital Models}
\label{sec:Human Capital Models}
Taste models pioneered by \citet{Becker} and \citet{Arrow} assert that employers have a preference for White workers and therefore only employ minorities at a lower wage rate to compensate them for lower utility, known as individual discrimination. With perfectly competitive markets, firms that are not prejudiced will offer a \enquote{fair} wage until prejudiced employers are forced to exit the market.

Even if direct wage discrimination does not occur, firms that are prejudiced may only offer minorities jobs in lower-paying occupations. Therefore, ethnic minorities may be forced into a lower-wage equilibrium if they are less likely to be offered higher-wage jobs \citep{Coate}.

Discrimination may therefore exist in practice if firms can collude to allow prejudice to exist long-term, especially if institutions permit this collusion \citep{LundbergB}. Alternatively, if customers are prejudiced and value the output of minorities less, this will lead to a material productivity difference, meaning economic incentives exist for firms to offer them lower wages (systemic discrimination).

\subsection{Statistical Discrimination and Signalling Models}
\label{sec:Stat_discrim and signalling}
Statistical discrimination occurs in the presence of asymmetric information. If employers cannot tell the level of productivity of workers, they may have an irrational perception that minorities are less productive. Consequently, they will only offer minorities a lower wage, regardless of their true productivity (which is unknown to the employer), causing ethnic minorities to be trapped in a low wage equilibrium. This is a self-fulfilling phenomenon in which discrimination can occur even when minorities are not actually of lower productivity, but are only offered lower wages because employers perceive them to be less productive \citep{Arrow} and \citep{LundbergB}. Linked to this is the concept of signalling models explored in \citet{Barr}, whereby two Nash equilibria can occur, due to an inability for minorities to convince employers of their equal productivity. This is also known as institutional discrimination.

Statistical discrmination can become entrenched as minorities make investment decisions regarding their education. If minorities expect to earn less in the labour market, their returns to education, and therefore incentives to invest in education, are lower than is economically efficient. This is also a self-fulfilling prophecy; incentives cause minorities to have less education than their White counterparts, and the expectations of the prejudiced employers are realised when they would not have been otherwise. The theory therefore predicts that ethnic minorities (and women) will earn less than White British men because of this endogeneity of investment in education. Over time, increased under-investment in education, combined with specialisation in other skills, can lead to comparative advantage in certain job types, further entrenching occupational segregation \citep{Altonji}.

Differences in preferences can also determine the occupational distribution of workers \citep{Thaler}. For example, if the preferences of White men are such that they are willing to work in recycling centres, when other ethnicities are much less willing to, then the scarcity of labour supply will command a higher-wage \citep{Altonji}. Aggregating these micro-economic factors leads to occupational clustering and wage gaps.

Finally, sociological factors can affect labour market outcomes. Family structures can vary by ethnicity due to societal norms, and lead to increased time out of the workforce, for mothers in particular. Difficulties in integrating into society may be added to by alienation, possibly due to past experiences of discrimination \citep{Berthoud}. These factors can hamper employment prospects and outcomes in the labour market.

\subsection{Systemic Discrimination}
\label{sec:systemic_discrim}
Systemic discrimination is where society as values the contribution of minorities less than for White workers. For example, if the public value the services of a White waiter more than the services of an ethnic minority waiter, then it provides a material difference in the \acrfull{mrp} of the workers. The firm consequently has an economic incentive to pay the minority a lower wage, even if it is based on irrational value judgements by society. These complex factors are difficult to measure and are not within the scope of this paper.

\section{Wage Determination}
\textcolor{red}{ %move this to the data section
Aside from discrimination, there are multiple factors which affect the wages of workers. Women may earn less if they expect to spend less time in the labour market, investing less in education. Further, those who are married may command a higher-wage, as those who are more productive may be more attractive in the marriage market \textcolor{red}{Chicago Rationale}). Additional complex effects of marriage on the labour market are explored by \citep{Ahituv}. 

Further, experience in the labour market (and the occupation or industry in particular) increases productivity and increases wages. Conversely, periods out of the labour market, through unemployment or maternal/paternal reasons will have a detrimental impact on wages. Skills can depreciate over time, and employers may offer a lower wage upon return. Those who invest in higher education are rewarded by employers due to increased productivity \citep{Becker}

Practical considerations can also affect wages; those living in specific regions (such as London) may naturally earn a higher-wage to compensate for increased living costs. Additionally, part-time workers may be willing to work for a lower wage if it is supplementing alternative forms of income. Increasing consumer prices for the entirety of the postwar period \citep{WorldBank} has also led to wage rises. Finally, those who are disabled may be subject to similar outcomes to ethnic minorities. Additional factors such as motivation and ambition also determine wages, but cannot be measured easily and are also probably endogenous (similar to the discussion in Section \ref{sec:Stat_discrim and signalling}). 
}

\ifstandalone
\bibliography{essaybib}
\fi
\end{document}