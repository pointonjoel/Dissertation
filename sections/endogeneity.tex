\documentclass[class=article, crop=false]{standalone}
\usepackage[subpreambles=true]{standalone}
\usepackage{import}
\usepackage{preamble}
\usepackage{pdfpages}
\begin{document}
\section{Endogeneity}
\label{sec:endogeneity}
The above decompositions focus primarily on outright pay discrimination. However, we have seen that occupational clustering occurs affects most ethnic groupings, putting downwards pressure on the pay of males (but not females). Occupational clustering may be due to differences in preferences, as explored in Section \ref{sec:Stat_discrim and signalling}; if minorities prefer not to work in particular industries, then we would expect to see a relatively higher wage premium for minorities working in such industries. However, observing Figures \ref{fig:m_wpremium-percnonwhite} and \ref{fig:f_wpremium-percnonwhite} for males and females respectively, there is very little correlation (if any) between the percentage of non-White workers and the wage premium in each industry. This implies preferences are not causing occupational clustering.

%\begin{comment}
\import{sections/}{cross_correls}
%\end{comment}
%
%Therefore, it would be useful to explore whether the clustering is indicative of other discriminatory effects, namely institutional and statistical discrimination, as discussed in Section \ref{sec:Stat_discrim and signalling}.

An alternative factor causing clustering may be perceptions. If workers believe that an industry is particularly \enquote{racist}, they may avoid it even if this leads to lower remuneration (which may explain the negative effect of clustering on the wages of males seen in Section \ref{sec:Males}). Minorities may also choose to avoid such industries which have historically undervalued, or currently undervalue, the productivity of minorities. However, this is not seen in Figures \ref{fig:m_wpremium-percracist} and \ref{fig:f_wpremium-percracist}, with no correlation between racist attitudes and the wage premium; those working in \enquote{racist} industries do not seem to experience a larger (negative) wage premium. Furthermore, there is no significant correlation between racist attitudes and the proportion of non-White workers in each industry (Figures \ref{fig:m_percnonwhite-percracist} and \ref{fig:racist_nonwhite}), meaning that minorities do not seem to be avoiding \enquote{racist} industries.

Two factors can explain this. Firstly, the survey of \enquote{British-ness} attitudes from the \acrlong{us} dataset could be a poor indicator of racist attitudes. Levels of reported racist incidents are available but suffer from a poor sample size. Thus, analysis using a more accurate proxy may yield alternative results. Secondly, the data uses industry, rather than occupation, data due to the incompatibility of 1992 and 2007 occupational codes. Finally, since Figures \ref{fig:explained_male_stacked} and \ref{fig:explained_female_stacked} outlined that education was a positive factor influencing the wages of all minorities, both male and female, it could be that male minorities have high levels of education yet employers undervalue their quality/productivity.

Therefore, the dataset provides no clear indication of why occupational clustering occurs. Given the importance of occupational clustering in determining wages, as discussed in Section \ref{sec:explained_diff}, future research into this will provide valuable insights into how policy can address the pay gap, particularly for males.

%"Could be that the proxy is bad. Or that the barrier is not racism, something else. Could be perceptions of their quality."
%\import{graphs/}{wpremium-percracist}


\end{document}