\documentclass[class=article, crop=false]{standalone}
\usepackage[subpreambles=true]{standalone}
\usepackage{import}
\usepackage{preamble}
\usepackage{pdfpages}
\begin{document}
\begin{abstract}
This paper uses \acrlong{lfs} data to study the ethnic pay gap from 1997 to 2019, pooling 23 years of data to create one of the largest analyses of the UK Labour Market. Decomposition results show vastly different outcomes by ethnic group and gender. Black, Asian and mixed race males find themselves earning a lower wage than their White counterparts, mainly due to occupational clustering. Only Chinese males do not experience a wage gap, in part due to higher levels of education. All ethnic minority female workers, along with Chinese and mixed race males, have no evidence of outright pay discrimination. Black and Asian workers have around 50\% of their wage gap unexplained. Both male and female minorities have higher levels of education, yet cluster into occupations, which is possible evidence of \textcolor{red}{structural and statistical discrimination}. Evidence of discrimination has been stationary over time and, if anything, worsened, even in the past 5 years.
\textcolor{red}{155 words}

\vspace{100pt}
\centering
Thanks to Dr Olena Nizalova and 
Guillermo Cabanillas-Jiménez for their support in overcoming the complex Econometric procedures through numerous consultation sessions. Special thanks to Dr Luke Buchanan-Hodgman for continued support over the past three years, and guidance with choosing a dissertation topic - it has made a world of difference.
\end{abstract}
\end{document}