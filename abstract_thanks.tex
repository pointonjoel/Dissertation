\documentclass[class=article, crop=false]{standalone}
\usepackage[subpreambles=true]{standalone}
\usepackage{import}
\usepackage{preamble}
\usepackage{pdfpages}
\begin{document}
\thispagestyle{plain}
\begin{center}
   \textbf{Abstract} 
\end{center}
This paper uses \acrlong{lfs} data to study the ethnicity pay gap from 1997 to 2019, pooling 23 years of data to create one of the largest analyses of the UK Labour Market. Blinder–Oaxaca decompositions show that ethnic minority males face statistically significant levels of unexplained negative wage differences. Conversely, ethnic minority women earn a higher wage than White women, due to favourable characteristics, and sometimes despite unexplained negative wage pressure. Both male and female Asian workers, in particular, have evidence of  direct pay discrimination. All minorities of both sexes have higher levels of education, yet men cluster into lower-paid positions, but women into higher paid ones than their White counterparts. Evidence of discrimination has been stationary over time and, if anything, worsened, even in the past 5 years. Unexplained wage gaps seemed to widen immediately following the financial crisis, although volatility prevents reliable conclusions from being made.

%Decomposition results show vastly different outcomes by ethnic group and gender. Black, Asian and mixed race males find themselves earning a lower wage than their White counterparts, mainly due to occupational clustering. Only Chinese males do not experience a wage gap, in part due to higher levels of education. All ethnic minority female workers, along with Chinese and mixed race males, have no evidence of outright pay discrimination. Black and Asian workers have around 50\% of their wage gap unexplained. Both male and female minorities have higher levels of education, yet cluster into occupations, which is possible evidence of \textcolor{red}{structural and statistical discrimination}. Evidence of discrimination has been stationary over time and, if anything, worsened, even in the past 5 years.
%\textcolor{red}{150 words}

\vspace{50pt}
%\centering
\begin{center}
   \textbf{Acknowledgements}
\end{center}
\linebreak
Many thanks to Dr Amanda Gosling for giving her support and expertise in writing this paper, and fostering my interest in Economics; without you, this dissertation would not have been possible. Thanks to Dr Olena Nizalova and Guillermo Cabanillas-Jiménez for their support in overcoming the complex Econometric procedures through numerous consultation sessions. Special thanks to Dr Luke Buchanan-Hodgman for the continued support over the past three years, and guidance with choosing a dissertation topic --- it has made a world of difference. Thank you to Steve Sanders for special access to a high-specification computer with Stata/MP, without which the analysis would not have been possible. You have all gone above and beyond to support me, and it is very much appreciated.
%\vspace{-50pt}
\end{document}